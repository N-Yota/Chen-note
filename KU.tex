\documentclass{ltjsarticle}

\usepackage{luatexja}
\usepackage{graphicx}
\usepackage[colorlinks=true,linkcolor=blue,citecolor=blue,urlcolor=blue]{hyperref}
\usepackage{xcolor}
\usepackage{mathtools}
\usepackage{siunitx}
\usepackage{amsmath}
\usepackage{amssymb}
\usepackage{sepnum} % sepnumパッケージを読み込む
\usepackage{at}
\usepackage{bm}
\usepackage{cancel}
\usepackage{float} % ← プリアンブルに追加
\usepackage{titlesec}
\numberwithin{equation}{section} % 数式番号をセクションごとに分ける
\titleformat{\subsubsection}
  {\normalfont\large\bfseries}
  {\thesubsubsection}{1em}{}

\newcommand{\underlab}[3][.35ex]{%
  \mathrel{\mathop{#2}\limits_{\mathclap{\lower#1\hbox{\text{\footnotesize\textcircled{\scriptsize #3}}}}}}%
}



\usepackage[normalem]{ulem} % \uline, \uwave

% 直線 → \ulineunder、波線 → \uwaveunder
\newcommand{\ulineunder}[2]{%
  \underset{\smash{\text{\scriptsize #2}}}{\uline{\ensuremath{#1}}}%
}
\newcommand{\uwaveunder}[2]{%
  \underset{\smash{\text{\scriptsize #2}}}{\uwave{\ensuremath{#1}}}%
}

% 下線の“下げ量”と太さ(グローバル設定)
\setlength{\ULdepth}{1.2ex} 
\renewcommand\ULthickness{0.8pt} % 波線/直線の太さ


\usepackage[most]{tcolorbox}
\newtcolorbox{eqbox}[1]{%
  enhanced,
  colback=white, colframe=black,
  boxrule=0.4pt, arc=6pt,
  left=8pt, right=8pt, top=8pt, bottom=8pt, % 余白
  title={#1},                 % ← 引数がタイトルになる
  fonttitle=\bfseries,        % タイトルの体裁
  coltitle=black,
  colbacktitle=gray!10,       % タイトル帯の背景色(要らなければ white)
  attach boxed title to top left={yshift=-2mm, xshift=6pt}, % 箱内の左上に配置
  boxed title style={sharp corners, boxrule=0pt}, % タイトル枠線なし
}


\title{Introduction to Plasma Physics}
\author{中 陽太}
\date{2025年8月29日}

\begin{document}

\maketitle

% 目次
\tableofcontents

\section*{前書き}
 以下の内容は、主に

\cite{a} Francis F. Chen, "Introduction to Plasma Physics and Controlled Fusion"\par
\cite{b} 宮本健郎 『プラズマ物理の基礎』\par
\cite{c} EMANの物理学\\
を参照している。





\section{プラズマとは} \label{section1}
\subsection{自然界でのプラズマ}
 物質の温度を上げていくと、固体、液体、気体と相変化し、さらに温度を上げると、気体分子は原子から電子が剥ぎ取られ、プラスの電荷を持ったイオンと負の電荷を持った電子とが混じった気体になる。このような高温の電離ガス状態を\textcolor{red}{プラズマ}という。
固体、液体、気体と並んで、「第四の状態」と呼ばれることもある。

 一般的に、プラズマは真空中にしか存在しない。なぜなら、空気がプラズマを冷やし、イオンと電子を再び結合させるからである。自然界においては、宇宙空間は密度がかなり低いが一種のプラズマだし、
点灯中の蛍光灯の中やロウソクの炎などは全ての原子から電子が剥がれるほどではないがこれも一種のプラズマだし、さらにオーロラや電離層もプラズマだし、冬場に体験する静電気の放電や、溶接工事に使う電気放電、落雷などもプラズマである。太陽もまたプラズマの塊である。実に、この宇宙の99\%以上の場所がプラズマであって、地球のようなそうでない場所の方が珍しいくらいである。

 Saha方程式では、熱平衡状態にある気体において、イオン化の割合を教えてくれる。

\begin{eqbox}{Saha方程式}
\begin{equation}
\frac{n_i}{n_n} \approx 2.4 \times 10^{21} \frac{T^{3/2}}{n_i} e^{-U_i/KT} 
\end{equation}  
\end{eqbox}

ここで$n_i$はイオン化された原子の密度$[m^{-3}]$、$n_n$は原子の密度であり、$K$はボルツマン定数、$U_i$は気体のイオン化エネルギーである。

(1.1)式から、$T$が大きくなるにつれて$n_i$の割合が増える、つまり気体がプラズマ状態になることがいえる。これがプラズマが高温でのみ存在する理由である。

Saha方程式の物理的な意味を指摘すると、気体中の原子は熱エネルギーに分布があり、たまたま十分大きいエネルギーの衝突を受けると電子が飛び出して電離する。低温ではそのような“速い原子”は少なく、電離はまれ。Saha式の 
$\exp{(-U_i/KT)}$は「速い原子の数が温度に対して指数関数的に減る(=低温で急減)」ことを表す。また、いったん電離しても、電子と出会えば再結合して中性に戻る。つまり、再結合率は電子密度に依存する。これが$n_i ^{-1}$となる要因である。星間ガスでは 
$n_i \sim 1 cm^{-3}$と非常に低密度のため、再結合が遅く、高温でなくてもプラズマ状態が保たれやすい。


\subsection{プラズマの定義}
いかなるイオン化した気体をプラズマと呼ぶわけではない。プラズマの定義は次のように表される。
\begin{center}
プラズマとは、荷電粒子と中性粒子からなる準中性ガスであり、集団運動を示すものである。  
\end{center}

準中性ガスの説明は1.4節で説明する。集団運動について、これはプラズマ粒子間に働くクーロン力が遠距離まで作用することに起因している。クーロン力は$1/r^2$で減衰するのに対し、プラズマの場合、立体角を考慮すると、Aに対するBの体積は$r^3$に比例する。
そのため、遠距離においてもプラズマは影響し合っている。つまり、集団運動とは、局所的な条件だけでなく、遠隔領域におけるプラズマの状態にも依存する運動を指す。実際、これらの性質がプラズマ物理学という研究分野を豊かにしている。

\begin{figure}[htbp]
  \centering
  \includegraphics[width=0.7\linewidth]{solid_angle.png}
  \caption{遠距離に作用するクーロン力}
  \label{fig:sample}
\end{figure}


\subsection{温度の概念} \label{section1.3}
先に進む前に、温度の物理的な概念を見直そう。熱平衡状態の気体の速度分布はマクスウェル分布に従う。

\begin{eqbox}{一次元のマクスウェル分布}
\begin{equation}
f(u) = A \exp\left(-\frac{1}{2}mu^2/KT\right),\hspace{2pc} A = n\left(\frac{m}{2\pi KT}\right)^{1/2} \label{Max}
\end{equation}
\end{eqbox}

マクスウェル分布の導出は\url{https://www1.doshisha.ac.jp/~bukka/lecture/statistic/pdftext/std-02.pdf}を参照してほしい。

分布の幅は定数Tによって特徴づけられ、これを温度と呼ぶ。Tの正確な意味を理解するために、この分布における粒子の平均運動エネルギーを計算しよう。

\begin{equation}
 E_{av} = \frac{\int_{-\infty}^{\infty} \frac{1}{2}mu^2 f(u)du}{\int_{-\infty}^{\infty} f(u)du} \label{AV.K}
\end{equation}
ここで、
\begin{equation}
  v_{th} \coloneqq (2KT/m)^{1/2}, \hspace{1pc} y \coloneqq u/v_{th} \label{df.v_th}
\end{equation}
と定義すると、式\eqref{Max}は

\begin{center}
 $f(u) = A \exp(-u^2/v^2_{th})$ 
\end{center}
となり、式\eqref{AV.K}は

\begin{center}
  $E_{av} = \dfrac{\frac{1}{2}mAv^3_{th}\int_{-\infty}^{\infty}[\exp(-y^2)]y^2dy}{Av_{th}\int_{-\infty}^{\infty}[\exp(-y^2)]dy}$
\end{center}
となる。分子における積分部分について、部分積分を用いることで、

\begin{center}
  $\int_{-\infty}^{\infty} [-\frac{1}{2}\exp(-y^2)]'\cdot ydy = \left[-\frac{1}{2}\exp(-y^2)y\right]_{-\infty}^{\infty} - \int_{-\infty}^{\infty} -\frac{1}{2}\exp(-y^2)dy
  = \frac{1}{2}\int_{-\infty}^{\infty}\exp(-y^2)dy $
\end{center}
となり、積分部分が相殺される。これより、

\begin{equation}
  E_{av} = \frac{\frac{1}{2}mAv_{th}^3\frac{1}{2}}{Av_{th}} = \frac{1}{4}mv_{th}^2 = \frac{1}{2}KT \label{av.kenetic}
\end{equation}
が得られる。つまり、平均運動エネルギーは$\frac{1}{2}KT$である。

3次元への拡張は簡単であるため、導出を省き、結果を書くと、
\begin{equation}
  E_{av} = \frac{3}{2}KT
\end{equation}

上で見たように、$T$は$E_{av}$で記述できるため、プラズマ物理では温度をエネルギーの単位で表すことが慣習になっている。次元による混乱を避けるため$E_{av}$ではなく、$KT$で温度を表す。$KT = 1\space eV = 1.6 \time 10^{-19}\space J $から

\begin{center}
  $T= \dfrac{1.6\time 10{-19}}{1.38\time 10^{-23}} = {11,600}$
\end{center}
したがって、変換係数は、

\begin{equation}
  1\;\unit{eV}= {11,600}\; \unit{\kelvin}
\end{equation}
である。

面白いのは、プラズマは異なる温度を持つことである。電子とイオンは衝突頻度の違いにより、それぞれ異なる温度
  \( T_e, T_i \) を持つことがある。磁場中では、一種類の粒子でもローレンツ力の作用により、
  平行方向の温度 \( T_{\parallel} \) と垂直方向の温度 \( T_{\perp} \) が異なる場合がある。

次章に進む前に温度の概念の誤解を解かなければならない。高温であることが必ずしも「大量の熱」を意味するわけではないのだ。
例えば、蛍光灯内の電子温度は約 $2\times 10^4 \ \mathrm{K}$ だが、電子密度が低いため壁への熱伝達は小さい。
タバコの灰は高温でも、含まれる熱量が少ないため手に落ちても大きな火傷をしない。実験室プラズマでは $10^6 \ \mathrm{K}$ 程度の温度を持つこともあるが、
密度が $10^{18}$--$10^{19} \ \mathrm{m^{-3}}$ と低いため、壁加熱は深刻な問題とならない。


\subsection{デバイ遮蔽}
プラズマが示す特徴的な挙動は電位を遮蔽することである。図\refeq{fig.debye}のように電源に繋がれた2つの荷電粒子をプラズマ中に入れた状況を考える。それぞれの荷電粒子は逆符号の粒子を印加し、荷電粒子の周りにイオンまたは電子の「雲」が作られる。
荷電粒子のすぐ近くでは多くのイオン(電子)が引き寄せられるため、密度及びポテンシャルの勾配は高く、逆に遠方では低い。もしプラズマが低温で、熱運動がないのであれば、ポテンシャルは完全に遮蔽される。しかし、温度が高ければ、「雲」の
端では静電ポテンシャルが低いため、熱エネルギーがその壁を越えることができる。このときの「雲」の縁は、ポテンシャルエネルギーが粒子の熱エネルギーKTとほぼ等しくなる半径で生じ、遮蔽は完全ではない。
$KT/e$オーダーのポテンシャルがプラズマに漏れ込み、そこに有限の電場が存在することとなる。

\begin{figure}[htbp]
  \centering
  \includegraphics[width=0.7\linewidth]{debye.png}
  \caption{デバイ遮蔽の様子}
  \label{fig.debye}
\end{figure}


雲の中心(荷電粒子の位置)$x=0$のポテンシャルを$\phi _0$として、そこから減少し、無限遠で0になるポテンシャルを考える。このときの$\phi (x)$を求めてみよう。

一次元のポアソン方程式より、

\begin{equation}
  \epsilon_0 \nabla^2 \phi = \epsilon_0 \frac{d^2\phi}{dx^2} = -e(n_i - n_e) \hspace{1pc}  (Z=1) \label{poisson}
\end{equation}
\\
もし遠方の密度が$n_\infty$ であるなら、
\begin{center}
  $n_i = n_\infty$
\end{center}
となる。

ポテンシャルエネルギー$q\phi$の存在下において、電子の分布関数(ボルツマン分布)は、
\begin{equation}
  f(u) = A\exp[-\left(\frac{1}{2}mv^2 + q\phi\right)/KT_e]
\end{equation}
$q=-e$として、$u$で積分すると、

\begin{align}
  n_e &= \iiint A\exp[-\left(\frac{1}{2}mv^2 - e\phi\right)/KT_e]du^3 = \underline{A}e^{\frac{e\phi}{K T_e}} 
\underset{\equiv n_0}{\underline{\iiint e^{-\frac{m u^2}{2 K T_e}} \, du^3}}\\
       &= n_0 \exp(e\phi/KT_e)
\end{align}
遠方($x\to \infty$)で$\phi \to 0$より、
\begin{center}
  $n_e = n_0\cdot 1 = n_o = n_\infty$
\end{center}
これより、

\begin{center}
 $n_e =  n_\infty \exp(e\phi/KT_e)$
\end{center}
この式は3.5節でより物理的な洞察をもって導く。以上のことから、\eqref{poisson}式を書き直すと、

\[
\epsilon_0 \frac{d^2\phi}{dx^2}= en_\infty(e^{e\phi/KT_e} - 1)
\]
クーロンポテンシャルよりも運動エネルギーの方がはるかに大きい領域、つまり、$|e\phi / KT_e| \ll 1$の領域においてテーラー展開することで、
\[
 \epsilon_0 \frac{d^2\phi}{dx^2} = en_\infty\left[\frac{e\phi}{KT_e}+ \frac{1}{2}(\frac{e\phi}{KT_e})^2 + \cdots \right] \label{telor.1}
\]
となる。1次の項だけを考えると、

\begin{equation}
  \frac{d^2 \phi}{dx^2} = \frac{4\pi n_\infty e^2}{k_B T_e}\phi \label{debye.eq}
\end{equation}
を得る。ここでデバイ長を次のように定義する。

\begin{eqbox}{デバイ長}
\begin{equation}
 \lambda_D = \left(\dfrac{\epsilon_0 KT_e}{ne^2}\right)^{1/2} \label{debye}
\end{equation}
\end{eqbox}
これにより、式\eqref{debye.eq}の解は、
\begin{equation}
  \phi = \phi_0 \exp(-|x|/\lambda_D) \label{debye.poten}
\end{equation}
となる。\eqref{debye.poten}の表す意味は、試験電荷などといった外乱で生じた静電ポテンシャルが、距離とともに指数関数的に減衰し、その減衰の速さを決める長さが デバイ長 
$\lambda_D$\eqref{debye}ということである。具体的に、$1\lambda_D$離れると$\phi$は元の3\%、$3\lambda_D$で約5\%、$5\lambda_D$で1\%未満となる。それより十分大きなスケールではプラズマはほぼ中性に見える。
ゆえに、デバイ長は遮蔽の距離を決める尺度となる。

密度が増えると$\lambda_D$が減るのは、より電子が集まり、雲が濃くなるからである。$\lambda_D$の定義に使われているのは電子の温度である。なぜなら、電子はイオンに比べはるかに軽く、動きやすいため、電子が動くことにより遮蔽が生じるからである。

式\eqref{debye}の便利な式として、
\begin{align}
  \begin{split}
    \lambda_D &= 69(T_e/n)^{1/2}\si{\metre}, \hspace{2pc}  T_e \; \text{in} \; \si{\kelvin}\\
    \lambda_D &= 7430(KT_e/n)^{1/2}\si{\metre}, \hspace{2pc} KT_e \; \text{in} \; \si{\electronvolt}
  \end{split}
\end{align}
を与えておく。

ここまでで先に述べた「準中性」を定義できる。もし距離$L$の系が$\lambda_D$よりはるかに大きい場合、デバイ遮蔽が生じる。その外では$\nabla^2 \phi$は非常に小さく、$n_i$は$10^{-6}$以上の精度で$n_e$と等しい。
つまり、プラズマの「準中性」とは、$n$をプラズマ密度と呼ばれる一般的な密度とした場合、$n_i \simeq n_e \simeq n$とみなせるほど十分に中性であるが、興味深い電磁気力がすべて消滅するほど中性ではない状態のことである。 

デバイ遮蔽は、電子が非常に高速で互いに衝突せず、熱分布を維持できない場合に破られる。後述するように、電子が非常に高温の場合、電子衝突は稀である。この場合、イオンの正電荷に引き寄せられた一部の電子は、角度をつけて高速で進入し、惑星を周回する衛星のようにイオンの周囲を軌道運動する。
この仕組みは、後述するラングミュアプローブの解説で明らかになる。この効果を「反遮蔽」と呼ぶ者もいる。

1つの種類の系でもデバイ遮蔽は起きることがいえるが、ここでは立ち入らないことにする。

\subsection{プラズマパラメータ}
上で述べたように、デバイ遮蔽が生じるのは「雲」を作るのに十分な粒子が存在するときである。
\eqref{debye.poten}式からデバイ領域の粒子数$N_D$を計算できる。
\begin{equation}
  N_D = n\cdot \frac{4}{3}\pi \lambda_D ^3 = 1.38 \time 10^6T^{3/2}/n^{1/2} \hspace{1pc}(T\text{in}\, \si{\kelvin})
\end{equation}

「集団運動」をするためには、$\lambda_D \ll L$ に加えて、
\begin{equation}
  N_D \lll 1
\end{equation}
が必要である。この$N_D$を\textcolor{red}{プラズマパラメータ}と呼ぶ。

\subsection{プラズマの条件}
イオン化された気体がプラズマとみなされる条件は次の3つである。
\begin{enumerate}
  \item $\lambda_D \ll L$
  \item $N_D \ggg 1$
  \item $\omega \tau > 1$
\end{enumerate}
1,2の条件はこれまでに述べた。3について、プラズマ中の荷電粒子は、電磁力によって支配されている必要がある。
しかし、弱電離ガス(例:飛行機のジェット排気)では、中性粒子との衝突頻度が高すぎるため、運動は流体力学的な力に支配される。
典型的なプラズマ振動の角振動数を$\omega$、中性粒子との平均衝突時間を$\tau$とすると、プラズマと言えるためには、$\omega \tau > 1$が必要である。

\section{1粒子の運動}
\subsection{一様な電場と磁場}
プラズマの解析の難しい点は、流体のように個々の運動は考慮せず、集団運動で考える時もあれば、個々の粒子の集まりと考える時もあることである。
まずは簡単な例として、この章では、1粒子が電場または磁場においての運動を確認しよう。

\subsubsection{E=0}
この系では粒子は単にサイクロトロン運動をする。運動方程式は、
\begin{equation}
  m\frac{d\bm{v}}{dt} = q\bm{v}\times \bm{B} \label{E=0}
\end{equation}
ここで、$\bm{B}$はz方向を向き、$\bm{B}=B\bm{\hat{z}}$である。\eqref{E=0}式から、
\begin{equation}
\begin{gathered}
  m\dot{v}_x = qB v_y, \quad
  m\dot{v}_y = -qB v_x, \quad
  m\dot{v}_z = 0 \\
  \begin{aligned}
    \ddot{v}_x &= \frac{qB}{m}\dot{v}_y = -\left(\frac{qB}{m}\right)^2 v_x \\
    \ddot{v}_y &= -\frac{qB}{m}\dot{v}_x = -\left(\frac{qB}{m}\right)^2 v_y \label{E=0.eq}
  \end{aligned}
\end{gathered}
\end{equation} 
これは調和振動子を表し、その振動数を

\begin{eqbox}{サイクロトロン振動数}
\begin{equation}
  \omega_c = \frac{|q|B}{m} \label{cyclon}
\end{equation}
\end{eqbox}  
と定義すると、\eqref{E=0.eq}式の解は、

\[
v_{x,y} = v_\perp \exp(\pm i\omega_ct + i\delta_{x,y})
\]
ここで、\pm の符号はイオンであれば「+」、電子であれば「−」である。また、$v_\perp$は\bm{B}に垂直な速度の成分である。

位相の任意性から、
\begin{subequations}
  \begin{equation}
    v_x = v_\perp e^{i\omega_c t} = \dot{x} \label{normal.x}
  \end{equation}
と選ぶことにより、
  \begin{equation}
    v_y = \frac{m}{qB}\dot{v_x} = \pm \frac{1}{\omega_c}\dot{v_x} = \pm iv_\perp e^{i\omega_c t} = \dot{y} \label{normal.y}
  \end{equation}
\end{subequations}
これらを積分することで、

\begin{equation}
  x - x_0 = -i\frac{v_\perp}{\omega_c}e^{i\omega_c t}, \hspace{2pc} y -y_0 = \pm \frac{v_\perp}{\omega_c}e^{i\omega_c t} \label{larmor.eq}
\end{equation}
ここで、ラーモア半径を

\begin{eqbox}{ラーモア半径}
\begin{equation}
  r_L \equiv \frac{v_\perp}{\omega_c} = \frac{mv_\perp}{|q|B} \label{larmor.radi}
\end{equation}
\end{eqbox}  
を定義すると、\eqref{larmor.eq}式は、

\begin{equation}
  x - x_0 = r_L \sin \omega_c t, \hspace{2pc} y -y_0 = \pm r_L \cos \omega_c t \label{guiding_center_eq}
\end{equation}
となる。これは案内中心$(x_0, y_0)$を中心とした円軌道を表す。

粒子の回転運動の向きは磁場を減らそうとする向きであり、イオンと電子で反対になる。また、この働きからプラズマは反磁性であると言える。
粒子はこのサイクロトロン運動に加え、\bm{B}の方向に$v_z$で等加速度運動している。

\subsubsection{有限の電場}
次に、一様な電場がある場合を考える。同様に\bm{B}の方向にz軸をとり、$E_y=0$となるように\bm{E}をx-z面に配置するように座標空間をとる。
先の議論からz成分と他の成分は分けて考えることができる。運動方程式は、

\begin{equation}
  m\frac{d\bm{v}}{dt} = q(\bm{E} + \bm{v}\times \bm{B}) \label{finite.E}
\end{equation}
z成分は簡単に求まり、

\begin{equation}
  \frac{dv_z}{dt} = \frac{q}{m}E_z \hspace{3pc} \therefore v_z = \frac{qE_z}{m}t + v_{z0}
\end{equation}
となる。

残りの成分、つまり\bm{B}に垂直な面での運動は、\eqref{finite.E}式から
\begin{align}
  \begin{split}
    \frac{dv_x}{dt} &= \frac{q}{m}E_x \pm \omega_c v_y\\
    \frac{dv_y}{dt} &= 0 \mp \omega_c v_x
  \end{split}
\end{align}
微分してそれぞれ計算すると、

\begin{subequations}
  \begin{align}
    \ddot{v}_x &= -\omega_c^2v_x\\
    \ddot{v}_y &= \mp \omega_c\left(\frac{q}{m}E_x \pm \omega_c v_y\right) = -\omega^2_c v_y -\omega^2_c\frac{E_x}{B} \label{v_e.eq}
  \end{align} \label{base_eq}
\end{subequations} 
  ここで、前節との類似点から粒子がサイクロトロン運動をすることを見越して、

\begin{equation}
  v_y = \uwaveunder{\pm v_\perp e^{i\omega_c t}}{サイクロトロン運動}+v_E \label{if.ve}
\end{equation}
を仮定する\footnote{Introduction to Plasma Physics and Controlled Fusionの方法ではなく、\url{https://youtu.be/6PL0fKNmRmQ}での方法に従った。}。これの2階微分は、

\begin{equation}
  \begin{aligned}
    \ddot{v}_y = \mp i\omega_c ^2 v_\perp e^{i\omega_c t} &= -\omega_c ^2 (\ulineunder{\pm iv_\perp e^{i\omega_c t}+v_E}{$=\,v_y$})+\omega_c ^2 v_E\\
                                                          &= -\omega_c ^2(-v_E + v_y)
  \end{aligned}
\end{equation}
これと\eqref{v_e.eq}式を比較すると、

\begin{equation}
  v_E = -\frac{E_x}{B}
\end{equation}
となる。これより、\eqref{if.ve}式を書き直すと
\begin{equation}
  v_y = \pm iv_\perp e^{i\omega_c t} - \frac{E_x}{B}
\end{equation}
さらに、
\begin{equation}
  v_x = \mp \frac{1}{\omega_c}\dot{v_y} = v_\perp e^{i\omega_c t}
\end{equation}
となる。サイクロトロン運動は以前と同様であるが、案内中心のy方向へのドリフト$v_E$が重ね合わされる。
$v_E$の一般式は、ベクトル形式で表すことができる。

\[
  \bm{E} \times \bm{B} = 
\begin{pmatrix}
E_x \\ 0 \\ E_z
\end{pmatrix}
\times
\begin{pmatrix}
0 \\ 0 \\ B
\end{pmatrix}
=
\begin{pmatrix}
0 \\ -BE_x \\ 0
\end{pmatrix}
\]
となることから、

\begin{equation}
  \boxed{\bm{v}_E = \frac{\bm{E}\times \bm{B}}{B^2}} \label{general.ve}
\end{equation}
と表せる。この$\bm{v}_E$は\textcolor{red}{$\bm{E}\times \bm{B}$ドリフト}と呼ばれる。$v_E$は$q$,$m$,$v_\perp$に依存しない。

このドリフトが生じる背景は図\refeq{fig.EB_drift}を見てほしい。まず半回転する間において、粒子は電場$E_x$により加速され、$r_L$が大きくなる。反対に、もう半回転する間は$r_L$は小さくなる。
この差が$-y$方向へのドリフトを生じさせる。イオンと電子では、$\bm{E}\times \bm{B}$ドリフトは同じ方向であるが回転は逆となる。

\begin{figure}[htbp]
  \centering
  \includegraphics[width=0.7\linewidth]{EB_drift.png}
  \caption{\bm{E}\times \bm{B}ドリフト\,+\,サイクロトロン運動の軌道}
  \label{fig.EB_drift}
\end{figure}

これらを踏まえて、z方向を含めた粒子の運動は、
\[
\bm{v} =
\left(
  \vcenter{\hbox{\shortstack{磁場に平行なクーロン力による\\等加速度直線運動}}}
\right)
+ \left(\text{サイクロトロン運動}\right)
+ \left(\text{\( \bm{E}\times\bm{B} \) ドリフト}\right)
\]
となる。もし$\bm{B}\perp \bm{E}$であるならば、第一項の運動はなくなる。

\subsubsection{重力場}
先の結果は$q\bm{E} \to \bm{F}$と置き換えることで一般的な力$\bm{F}$に適用できる。$\bm{F}$による案内中心のドリフトは\eqref{general.ve}式より

\begin{equation}
  \boxed{\bm{v}_f = \frac{1}{q}\frac{\bm{F}\times \bm{B}}{B^2}} \label{drift_F}
\end{equation}
特に、$\bm{F}$が重力$m\bm{g}$であるとき、

\begin{equation}
  \boxed{\bm{v}_g = \frac{m}{q}\frac{\bm{g}\times \bm{B}}{B^2}}
\end{equation}
となる。ここで注意すべきは、$\bm{v_E}$では粒子によらず同じ方向であったが、$\bm{v_g}$ではイオンと電子で方向が逆になる。
プラズマ中で電子とイオンが逆向きに動くことは電流を流れることを意味する。電子だけ動いても電流が流れると思うかもしれないが、
プラズマでは荷電分離がすぐに電場を生んで打ち消してしまうので、持続的に電流が流れるためには電子とイオンが逆向きに動く必要がある。
この$\bm{v}_g$により生じる電流密度は、

\begin{equation}
  \bm{j} = nev_{g,i} + n\cdot (-e)v_{g,e} = n(M+n)\frac{\bm{g}\times \bm{B}}{B^2}
\end{equation}
となる。

実際には、地上の重力$\bm{g}$は小さいため、$|\bm{v}_g|$は通常、無視してよい。しかし、磁力線が曲がっている場合、後述するように、遠心力に起因する「有効重力」が生じる。
遠心力は「重力」不安定と呼ばれるプラズマ不安定性の基盤であり、これは実際の重力とは無関係である。

\subsection{非一様磁場}
これまで見てきたように、一様な場では厳密な解を求めることができた。しかし、空間や時間によって変わる電場や磁場下での粒子の運動は非常に複雑であり、厳密解が得ることは難しい。
そこで近似解を求める方法として、非一様性を表すスケール長$L$を用いて、微小量$r_L/L$で展開することがよくなされる。この種の理論は「軌道理論」と呼ばれ、非常に複雑である。
ここでは、同時には1つの非一様性しか含まないような最も簡単な場合について考える。

\subsubsection{\nabla \bm{B} \perp \bm{B} : grad Bドリフト}
図\refeq{fig.gradB_drift}のようにz方向の磁場がy方向に応じて密度(大きさ)が増えるとする。
この場合の粒子の軌道は予想できる。上部と下部での磁場の大きさの違いによりラーモア半径が変わり、その差がドリフトを生み出す。($\bm{E}\times \bm{B}$ドリフトと同じ原理。)

1回転について平均したローレンツ力 $\bm{F} = q\bm{v} \times \bm{B}$について考える。$\bar{F_x}$について、図\refeq{fig.gradB_drift}からわかるように、下部および上部での回転それぞれで$F_x$はちょうど打ち消し合うため、$\bar{F_x}=0$である。

\begin{figure}[htbp]
  \centering
  \includegraphics[width=0.7\linewidth]{gradB_drift.png}
  \caption{非一様磁場でのドリフト}
  \label{fig.gradB_drift}
\end{figure}

$\bar{F_y}$は近似的に求めることになるが、ゆがみのない粒子軌道を考えることから出発する。ゆがみのない粒子軌道は、一様磁場で\eqref{normal.x},\eqref{normal.y},\eqref{larmor.eq}式とすでに求めた。
$r_L/L\ll 1$を満たす任意の場所において$\bm{B}$は1次までのテーラー展開で表せる。(2次以上は無視できるため。)

\begin{equation}
  \bm{B} = \bm{B}_0 + (\bm{r}\cdot \nabla)\bm{B} + \cdots
\end{equation}
\[
  B_z = B_0 + y(\partial B_z/\partial y) + \cdots
\]
展開点に$x_0=0, y_0=0$を選び、\eqref{normal.x}式で実部を取ると、

\begin{equation}
  F_y = -qv_x B_z(y) = -qv_\perp (\cos{\omega_ct})\left[B_0 \pm r_L(\cos{\omega_c t})\frac{\partial B}{\partial y}\right] \label{grad_B_drift}
\end{equation}
\eqref{grad_B_drift}式の第一項は1回転で平均すれば0となる。第二項は$\cos^2{\omega_c t}$の平均が1/2であることから、

\begin{equation}
  \bar{F_y} = \mp q v_\perp r_L \frac{1}{2}\frac{\partial B}{\partial y}
\end{equation}
これより、案内中心のドリフト速度は\eqref{drift_F}式より、

\begin{equation}
  \bm{v}_{gc} = \frac{1}{q}\frac{\bm{F} \times \bm{B}}{B^2} = \frac{1}{q} \frac{\bar{F_y}}{|B|}\hat{\bm{x}} = \mp \frac{v_\perp r_L}{B} \frac{1}{2} \frac{\partial B}{\partial y}\hat{\bm{x}}
\end{equation}
これには平均化した力をそのまま\eqref{drift_F}式に使っていいのかといった疑問を抱くかもしれないが、grad B ドリフトよりも旋回運動の方がずっと速いため、平均を考えた1周はドリフトにとっては一瞬であると考えられるので問題はない。
これをより一般的に書くと、$\bm{E}\times \bm{B}$ドリフトと同じ計算で、

\begin{equation}
  \boxed{\bm{v}_{\nabla B} = \pm \frac{1}{2}v_\perp r_L \frac{\bm{B}\times \nabla B}{B^2}} \label{df.grad_B_drift}
\end{equation}

この$\bm{v}_{\nabla B}$は\textcolor{red}{grad B ドリフト}と呼ばれる。$\pm$が表すようにイオンと電子で方向が反対になり、$\bm{B}$に垂直な方向に電流が生じる。
$\bm{v}_{\nabla B}$の正確な計算には、平均化プロセスにおいてドリフトを含む正確な軌道の使用が必要となる。

\subsubsection{湾曲磁場:曲率ドリフト}
磁力線が一定の曲率半径$R_c$で湾曲しており、|B|は一定である場合を考える(図\refeq{fig.curvature})。
粒子が熱運動で磁力線に沿って移動する際に、粒子が受ける遠心力により案内中心のドリフトが起きる。ランダムな速度の$\bm{B}$方向の2乗平均を$v_\parallel ^2$と表すと、
平均的な遠心力は

\begin{equation}
  \bm{F}_{cf} = \frac{mv_\parallel ^2}{R_c} \hat{\bm{r}} = mv_\parallel ^2 \frac{\bm{R}_c}{R_c ^2}
\end{equation}
これによって起きるドリフト速度は\eqref{drift_F}式より、

\begin{equation}
  \boxed{\bm{v}_R = \frac{1}{q}\frac{\bm{F}_{cf} \times \bm{B}}{B^2} = \frac{mv_\parallel ^2 }{qB^2} \frac{\bm{R}_c \times \bm{B}}{R_c ^2}}
\end{equation}
このドリフト$\bm{v}_R$は\textcolor{red}{曲率ドリフト}と呼ばれる。

\begin{figure}[htbp]
  \centering
  \includegraphics[width=0.7\linewidth]{curvature.png}
  \caption{湾曲磁場}
  \label{fig.curvature}
\end{figure}

湾曲磁場では、曲率ドリフトに加えてgrad B ドリフトも生じる。次にgrad B ドリフトを曲率半径方向の|B|の減少を考慮して計算してみよう。真空中ではマクスウェル方程式より、$\nabla \times \bm{B} = 0$が成り立っている。
図\refeq{fig.curvature} の円柱座標において、$\bm{B}$は$\theta$方向の成分のみ、$\nabla B$はr方向の成分のみであるため、$\nabla \times \bm{B}$はz成分のみとなる。
円柱座標系で表されたベクトル場$\bm{A}$の回転$\nabla \times \bm{A}$は、
\[
 \nabla \times \bm{A} = \left\{\frac{1}{r}\frac{\partial A_z}{\partial \theta}\right\}\bm{e}_r + \left\{\frac{\partial A_r}{\partial z}-\frac{\partial E_z}{\partial r}\right\}\bm{e}_\theta + \dfrac{1}{r}\left\{\frac{\partial}{\partial r}(rA_\theta) - \frac{\partial E_r}{\partial \theta}\right\}\bm{e}_z
\]
で表されるので、

\begin{equation}
  (\nabla \times \bm{B})_z = \frac{1}{r}\frac{\partial}{\partial r} (rB_\theta) = 0 \hspace{2pc} B_\theta \propto \frac{1}{r} \hspace{2pc} \therefore |B| \propto \frac{1}{R_c}
\end{equation}
また、
\[
  \nabla |B| = \frac{d|B|}{dr}\hat{r} \propto -\frac{1}{r^2}\hat{r} 
\]
より、
\begin{equation}
  \frac{\nabla |B|}{|B|} = -\frac{\bm{R}_c}{R_c ^2}
\end{equation}
これより、\eqref{df.grad_B_drift}式を用いて、

\begin{equation}
  \bm{v}_{\nabla B} = \mp \frac{1}{2}\frac{v_\perp r_L}{B^2}\bm{B}\times |B| \frac{\bm{R}_c}{R_c ^2} = \pm \frac{1}{2} \frac{v_\perp ^2}{\omega_c}\frac{\bm{R}_c \times \bm{B}}{R_c ^2B^2} = \frac{1}{2}\frac{m}{q}v_\perp ^2 \frac{\bm{R}_c \times \bm{B}}{R_c ^2 B^2}
\end{equation}

前の$\bm{v}_R$にこれを加えて、真空湾曲磁場中でのドリフトは
\begin{equation}
  \boxed{\bm{v}_R + \bm{v}_{\nabla B} = \frac{m}{q} \frac{\bm{R}_c \times \bm{B}}{R_c ^2 B^2}\left(v_\parallel ^2 + \frac{1}{2}v_\perp ^2\right)}
\end{equation}
となる。

このようなドリフトが加わるのは望ましくない。なぜなら、曲率ドリフト$\bm{v}_R$とgrad B ドリフト$\bm{v}_{\nabla B}$が 同じ垂直方向(イオンの場合、図\refeq{fig.curvature}では上向き)に出て加算されるため、案内中心は一方向に単調にずれていき、
最終的に外に出てしまう。核融合プラズマを閉じ込める目的で磁場をトーラス状にしても閉じ込められないのである。

 マクスウェル分布の場合、\eqref{av.kenetic}から、$v_\perp$が2自由度を含むことに注意すると、$\bar{v_\parallel ^2}$、$\frac{1}{2}\bar{v_\perp ^2}$は
 \begin{align*}
  \frac{1}{2}m\bar{v_\parallel ^2} &= \frac{1}{2}KT \hspace{2pc} \bar{v_\parallel ^2} = \frac{KT}{m}\\
  \frac{1}{2}m\bar{v_\perp ^2} &= KT \hspace{2pc} \frac{1}{2}\bar{v_\perp ^2} = \frac{KT}{m}
 \end{align*}
と表される。\eqref{df.v_th}を用いることで、平均曲率場ドリフトを次のように記述できる。

\addtocounter{equation}{-1}% 親番号を 2.30 に合わせる
\begin{subequations}
  \begin{equation}\label{eq:230a}
    \bar{\bm{v}}_{R+\nabla B} = \pm \frac{v_{th}^2}{R_c \omega_c}\hat{\bm{y}} = \pm \frac{\bar{r_L}}{R_c}v_{th} \bar{\bm{y}}
  \end{equation}
\end{subequations}
ここで、$\bar{\bm{y}}$は$\bm{R}_c \times \bm{B}$の向きである。この式から$\bar{\bm{v}}_{R+\nabla B}$は粒子の電荷には依存するが、質量には依存しないことがいえる。

\subsubsection{$\nabla B \parallel \bm{B}$ : 磁気ミラー}
磁場のだいたいの方向がz方向で、z方向に強さの変化する磁場を考える。磁場は軸対称であるとすると、$B_\theta = 0$、$\partial/\partial \theta = 0$である。(図\refeq{fig.mirror})
このような磁場が粒子を捕捉できるような力を生み出すことを以下に示そう。

\begin{figure}[htbp]
  \centering
  \includegraphics[width=0.7\linewidth]{magnetic_mirror.png}
  \caption{ミラー磁場中での粒子のドリフト}
  \label{fig.mirror}
\end{figure}

$\nabla \cdot \bm{B} = 0$より、

\begin{equation}
  \frac{1}{r}\frac{\partial}{\partial r}(rB_r) + \frac{\partial B_z}{\partial z} = 0
\end{equation}
ここで、円柱座標系で表されたベクトル場$\bm{A}$の発散が
\[
  \nabla \cdot \bm{A} = \frac{1}{r}\frac{\partial}{\partial r}(rA_r) + \frac{1}{r}\frac{\partial A_\theta}{\partial \theta} + \frac{\partial A_z}{\partial z}
\]
で表せることを用いた。
$\frac{\partial B_z}{\partial z}$の値が$r=0$で与えられ、$r$とともにあまり大きく変化しないとすると、近似的に、

\begin{equation}
  \begin{gathered}
    rB_r = -\int_{0}^{r} r\frac{\partial B_z}{\partial z}dr \simeq -\frac{1}{2}r^2\left[\frac{\partial B_z}{\partial z}\right]_{r=0}\\
    B_r = -\frac{1}{2}r\left[\frac{\partial B_z}{\partial z}\right]_{r=0}    
  \end{gathered} \label{term1234}
\end{equation}
となる。$r$による$|\bm{B}|$の変化はgrad B ドリフトを引き起こすが、$\partial B/\partial \theta = 0$なので、$\nabla B$に$\theta$方向の成分は存在しない。よって、$\bm{B}\times \nabla B$のr方向成分は0であるから、半径方向にはgrad B ドリフトは存在しない。
ローレンツ力の成分は

\begin{align}
  \begin{split}
    F_r     &= q(\underlab{v_\theta B_z}{1}-v_z \cancel{B_\theta})\\
    F_\theta&= q(\underlab{-v_r B_z}{2}+\underlab{v_z B_r}{3})\\
    F_z     &= q(v_r \cancel{B_\theta}-\underlab{v_\theta B_r}{4})
  \end{split}
\end{align}

$B_\theta = 0$より2つの項が消える。➀項と➁項は通常のラーモア運動を引き起こす項である。➂項は軸上では消え、軸から離れたところでは、

\[
  \bm{F} \times \bm{B} = (F_\theta \hat{\theta}) \times (B_r \hat{r}+B_z \hat{z}) = F_\theta B_z \hat{r} - F_\theta B_r \hat{z}
\]
ここで、$B_z \gg B_r$の近似を用いると、
\[
 \bm{F} \times \bm{B} = F_\theta B_z \hat{r} = qv_zB_rB_z\hat{r}
\]
これをドリフトの一般式に代入すると、
\begin{equation}
  \bm{v}_{d} = \frac{\bm{F} \times \bm{B}}{qB^2} = \frac{qv_zB_rB_z}{q(B_z^2+B_r^2)}\hat{r} \simeq v_z \frac{B_r}{B_z}\hat{r} \hspace{2pc} (B_z \gg B_r) \label{v_d,r}
\end{equation}
となり、r方向のドリフトを引き起こすことがわかる。

次に、磁力線の幾何から磁力線に沿う運動の必要条件を求めてみる。磁力線に沿った弧長の微小距離を$ds$とすると、$ds$は$\bm{B}$に平行なので、

\[
 \frac{dr}{ds} = \frac{B_r}{B}, \qquad \frac{dz}{ds} = \frac{B_z}{B} \hspace{1pc} \to \hspace{1pc} \frac{dr}{dz} = \frac{B_r}{B_z} 
\]
が成り立つ。これより$r$の変化率は、
\begin{equation}
  \frac{dr}{dt} = \frac{dr}{dz}\frac{dz}{dt} = \frac{B_r}{B_z}v_z \label{r_rate}
\end{equation}
となる。\eqref{v_d,r}式と\eqref{r_rate}式を比較すると完全に一致していることがわかる。つまり、ドリフト運動で動くr方向の速度と磁力線のr方向の変化率は一致する。
これはドリフト運動が磁力線に沿って移動していることを表す。長くなったが、以上をまとめると、➂項の力が$r$方向のドリフトを引き起こすが、このドリフトは単に案内中心を磁力線に沿わせるように働く。

結局、注目すべきは➃項である。\eqref{term1234}式より、

\begin{equation}
  F_z = \frac{1}{2}qv_\theta r(\partial B_z/\partial z)
\end{equation}
となる。1周にわたって平均をとるが、簡単のため、案内中心が軸上にあるような粒子を考える。このとき、$v_\theta$は回転の間、一定であり、qが+のときは時計回り、-のときは反時計回りであることを考えると、
$v_\theta$は$\mp v_\perp$と書ける。$r=r_L$から、平均的な力は

\begin{equation}
  \bar{F_z} = \mp \frac{1}{2}qv_\perp r_L\frac{\partial B_z}{\partial z} = \mp \frac{1}{2}q\frac{v_\perp ^2}{\omega_c}\frac{\partial B_z}{\partial z} = -\frac{1}{2}\frac{mv_\perp ^2}{B}\frac{\partial B_z}{\partial z}
\end{equation}
となる。
回転粒子の磁気モーメントを
\begin{eqbox}{磁気モーメント}
\begin{equation}
 \mu \equiv \dfrac{\frac{1}{2}mv_\perp ^2 }{B}  \label{df.mag_moment}
\end{equation}
\end{eqbox}
で定義すると

\begin{equation}
  \bar{F_z} = -\mu (\partial B_z / \partial z)
\end{equation}
となる。一般的には、

\begin{equation}
  \bm{F}_\parallel = -\mu (\partial B /\partial s) = -\mu \nabla_\parallel B \label{F.mag_moment}
\end{equation}
と書かれる。\eqref{df.mag_moment}式の定義が、面積$A$、電流$I$の円電流に関する磁気モーメントの定義$\mu= IA$と等価であることに注目してほしい。
1価に帯電したイオンの場合、電荷$e$が1秒間に$\omega_c/2\pi$回回転することにより電流$I$が発生する。$I=e\omega_c/2\pi$、$A=\pi r_L ^2 = \pi v_\perp ^2/\omega_c ^2$より

\[
 \mu = \frac{\pi v_\perp ^2}{\omega_c ^2}\frac{e\omega_c}{2\pi} = \frac{1}{2}\frac{mv_\perp ^2}{B}
\]
となる。

$\mu$が不変量であることを証明しよう。運動方程式の$\bm{B}$に沿った成分は\eqref{F.mag_moment}式を用いて

\begin{equation}
  m\frac{dv_\parallel}{dt} = -\mu \frac{\partial B}{\partial s}
\end{equation}
左辺に$v_\parallel$を、右辺にそれと同値の$ds/dt$をかけると、

\begin{equation}
  mv_\parallel \frac{dv_\parallel}{dt} = \frac{d}{dt}\left(\frac{1}{2}mv_\parallel ^2\right) = -\mu \frac{\partial B}{\partial s}\frac{ds}{dt} = -\mu \frac{dB}{dt} \label{time_shift_B}
\end{equation}
ここで、$dB/dt$は粒子から見た$B$の変化であり、$B$自身は時間に対して変化していない。粒子のエネルギーは保存しているはずだから、

\begin{equation}
  \frac{d}{dt}\left(\frac{1}{2}mv_\parallel ^2 + \frac{1}{2}mv_\perp ^2\right) = \frac{d}{dt}\left(\frac{1}{2}mv_\parallel ^2 + \mu B\right) = 0
\end{equation}
他にローレンツ力も働いているが仕事をしないので、考えるエネルギーは運動エネルギーのみで良い。\eqref{time_shift_B}式から

\[
 -\mu \frac{dB}{dt} + \frac{d}{dt}(\mu B) = 0
\]
したがって、
\begin{equation}
  d\mu/dt = 0
\end{equation}
が証明できた。

$\mu$の不変性は、初期のプラズマ閉じ込め計画の1つであった「磁気ミラー」の基礎である。熱運動によって粒子が弱磁場から強磁場へ動いたとき、$B$が増えるため、$\mu$を一定に保つために$v_\perp$も増える。
全運動エネルギーは一定なので、このとき必然的に$v_\parallel$は減少する。もし$B$がミラーの「のど」の部分で十分に強ければ、$v_\parallel$は最終的に0となり、弱磁場の方へ反射される。

このような状況は2つのコイルさえあれば実現できる(図\refeq{fig.trap})。2つのコイルによってできる非一様磁場は2つの磁気ミラーを形成し、その間にプラズマを閉じ込めることができる。この効果はイオンにも電子にも働く。

\begin{figure}[htbp]
  \centering
  \includegraphics[width=0.7\linewidth]{trapped.png}
  \caption{磁気ミラー間に閉じ込められたプラズマ}
  \label{fig.trap}
\end{figure}

しかし、この閉じ込めは完璧ではない。例えば、$v_\perp=0$の粒子は磁気モーメントを持たず、$\bm{B}$に沿った力を感じない。また、もし最大磁場である$B_m$が十分に大きくなかった場合、$v_\parallel$が0にならず、そのまま通過していくこともあるだろう。
では、どのような$B_0$ないしは$B_m$であれば粒子は逃げていってしまうのであろうか?中央の面での粒子の速度を$v_{\perp 0}$、$v_{\parallel 0}$、反射点での速度を$v_\perp ^\prime$、($v_\parallel =0$)とする。
さらに反射点での磁場を$B^\prime$とすると、$\mu$の不変性により

\begin{equation}
  \frac{1}{2}mv_{\perp 0} ^2 /B_0 = \frac{1}{2}mv_\perp ^{\prime 2}/B^\prime \label{mu.eq}
\end{equation}
また、エネルギー保存則から、

\begin{equation}
  v_\perp ^{\prime 2} = v_{\perp 0}^2 + v_{\parallel 0}^2 \equiv v_0 ^2 \label{v_parallel.eq}
\end{equation}
が成り立っている。\eqref{mu.eq}、\eqref{v_parallel.eq}式から、

\begin{equation}
  \frac{B_0}{B^\prime} = \frac{v_{\perp 0 ^2}}{v_\perp ^\prime 2} = \frac{v_{\perp0}^2}{v_0 ^2} \equiv \sin^2\theta \label{loss_cone.eq}
\end{equation}
となる。ここで定義した$\theta$は図\refeq{loss_cone}で表されるような$\bm{v}$と$v_\perp$からなる角度である。もし$\theta$が小さすぎると、$B^\prime$は$B_m$を超え、粒子は全くミラー効果を示さない。
\eqref{loss_cone.eq}式において$B_0$を$B_m$で置換すると、閉じ込められた粒子の最小$\theta$は次式で与えられる。

\begin{equation}
  \sin^2 \theta_m = B_0/B_m \equiv 1/R_m \label{mirror_ratio}
\end{equation}
$R_m$はミラー比率と呼ばれる。\eqref{mirror_ratio}式は速度空間における領域の境界を定義し、その形状は円錐状で、ロスコーンと呼ばれる(図\refeq{loss_cone})。
ロスコーン内に存在する粒子は閉じ込められていない。ロスコーンは$q$や$m$に依存しないことに注意してほしい。衝突がなければ、イオンも電子も等しくに閉じ込められる。衝突が発生すると、粒子は衝突でピッチ角を変えロスコーン内に散乱されることで失われる。
一般に、電子は衝突頻度が高いためより容易に損失する。

磁気ミラーは、宇宙線の加速メカニズムとしてFermiによって最初に提唱された。ミラー効果の例として、ヴァン・アレン帯における粒子の閉じ込めが挙げられる。
地球の磁場は極域で強く、赤道域で弱いため、かなり大きな半径$R_m$を持つ天然のミラーを形成する。

\begin{figure}[H]
  \centering
  \includegraphics[width=0.7\linewidth]{loss_cone.png}
  \caption{ロスコーン}
  \label{loss_cone}
\end{figure}


\subsection{非一様電場}
ここでは、磁場は一様であるが、電場は非一様であるような場を考える。簡単のため、$\bm{E}$はx方向を向き、x方向に正弦状に変化すると仮定する(図\refeq{nonuniform_E}):

\begin{equation}
  \bm{E} \equiv E_0(\cos kx)\hat{\bm{x}} \label{df.non}
\end{equation}

\begin{figure}[htbp]
  \centering
  \includegraphics[width=0.7\linewidth]{nonuniform_E.png}
  \caption{不均一電場中における回転粒子のドリフト}
  \label{nonuniform_E}
\end{figure}

 この電界分布は正弦波状の電荷分布により生じる。簡単のために設定した状況だが、実際、このような電荷分布は波動運動中のプラズマ内で生じうる。運動方程式は次の通りである。

 \begin{equation}
  m(d\bm{v}/dt) = q[\bm{E}(x) + \bm{v} \times \bm{B}] 
 \end{equation}
各成分は、

\begin{equation}
  \dot{v}_x = \frac{qB}{m}v_y + \frac{q}{m}E_x(x) \hspace{2pc} \dot{v}_y = -\frac{qB}{m}v_x 
\end{equation}
\begin{equation}
  \ddot{v}_x = -\omega_c ^2 v_x \pm \omega_c \frac{\dot{E}_x}{B} \label{ddot_v_x}
\end{equation}
\begin{equation}
  \ddot{v}_y = -\omega_c ^2 v_y - \omega_c ^2 \frac{E_x(x)}{B} \label{ddot_v_y}
\end{equation}
これを解くために、もし電場が弱ければ、近似として、擾乱されていない軌道を$E_x(x)$の評価に使用できる。電場が存在しない場合の軌道は\eqref{guiding_center_eq}式で与えられていた:

\begin{equation}
  x = x_0 +r_L \sin \omega_c t
\end{equation}
\eqref{df.non}、\eqref{ddot_v_y}式から、

\begin{equation}
  \ddot{v}_y = -\omega_c ^2 v_y - \omega_c ^2 \frac{E_0}{B} \cos k(x_0 + r_L \sin \omega_c t)
\end{equation}
このときの$v_y$は周波数$\omega_c$で起こる高速なサイクロトロン運動とゆっくりとしたドリフト運動の2つが含まれていると予想できる。我々の目標はドリフト速度$v_E$を求めることであるから、
サイクロトロン運動の振動成分を除去したい。そのため、1サイクルで平均を取る。

まず、$\bar{v}_x$について、\eqref{ddot_v_x}式において平均をとることで、$\overline{\ddot{v}}_x = 0$、$\bar{\dot{E}}_x = 0$から$\bar{v}_x = 0$がいえる。
$\overline{\ddot{v}}_x = 0$としたことに疑問に思うかもしれない。確かに、$v_x$はサイクロトロン運動に加えて、ドリフト運動にあたる速度を持つ。しかし、時間変化を考えたとき、ドリフト運動はサイクロトロン運動に比べて極めてゆっくりであるから無視できる。
それゆえ$\overline{\ddot{v}}_x$はサイクロトロン運動のみを考えてよく、もちろんその1サイクルの平均は0になる。同様に、$\overline{\ddot{v}}_y = 0$であるから、
\eqref{ddot_v_y}式において平均をとると、

\begin{equation}
  \overline{\ddot{v}}_y = 0 = -\omega_c ^2 \bar{v}_y - \omega_c^2\frac{E_0}{B}\overline{\cos k(x_0 + r_L \sin \omega_c t)}
\end{equation}
ラーモア半径は小さいので、$kr_L \ll 1$より、$\cos$及び$\sin$についてテーラー展開すると、

\begin{align}
  \cos k(x_0 +r_L \sin \omega_c t) &= \cos (kx_0) \cos (k_L \sin \omega_c t) - \sin (kx_0) \sin (kr_L \sin \omega_c t) \notag \\ 
                                   &\approx (\cos kx_0)(1-\frac{1}{2} k^2 r_L^2 \sin^2 \omega_c t) - (\sin kx_0)kr_L \sin \omega_c t
\end{align}
これに時間平均をとれば、最後の項は消え、

\begin{equation}
  \bar{v}_y = -\frac{E_0}{B}(\cos kx_0)\left(1-\frac{1}{4}k^2r_L ^2\right) = -\frac{E_x(x_0)}{B}\left(1-\frac{1}{4}k^2 r_L ^2\right)
\end{equation}
 したがって、通常の$\bm{E} \times \bm{B}$ドリフトは不均一性によって修正され、次のように表される。

 \begin{equation}
  \bm{v}_E = \frac{\bm{E} \times \bm{B}}{B^2}\left(1-\frac{1}{4}k^2r_L^2\right) \label{time_vary_E_cos}
 \end{equation}

 この物理的理由は容易に理解できる。通常の$\bm{E}\times \bm{B}$ドリフトではすべて$E_0$で一様であるのに対し、こちらのドリフトでは、正弦波分布において案内中心で$\bm{E}$が最大となるように仮定した。そのため、イオンは必ず$E_0$より弱い電場領域を通る。したがって、その平均ドリフトは案内中心で評価される$E/B$よりも小さくなる。
 線形に変化する電場中では、イオンは軌道の片側でより強い電場中におり、反対側では同量だけ弱い電場中にある。この場合、平均をとった際に$v_E$への補正項は打ち消し合う。ここから、補正項が$E$の2階微分関数に依存することが明らかである($\frac{1}{4}k^2r_L^2$項)。
正弦波分布の仮定から離れて、任意の$\bm{E}$の変動に対しては、$ik$を$\nabla$で置き換え、\eqref{time_vary_E_cos}式を次のように書けばよい。

\begin{equation}
  \bm{v}_E = \left(1+\frac{1}{4}r_L^2\nabla^2\right)\frac{\bm{E}\times \bm{B}}{B^2} \label{time_vary_general}
\end{equation}

イオンが感じる電場を考えると、ドリフト速度はラーモア半径$r_L=v_\perp/\omega_c$と電場変化スケール$k$に依存することは明らかである。\eqref{time_vary_general}式において、第2項は\textcolor{red}{有限ラーモア半径効果}と呼ばれる。 
この補正の意義を考えてみよう。$r_L$が電子よりもイオンの方がはるかに大きいため、イオンと電子のドリフト速度は異なる。この結果、電荷分離が起こり、新たな電場$E^\prime$を誘起する。もし、この$E^\prime$が元の電場を強める(正のフィードバック)ように働くと、$\bm{E}$は無限に増大し、プラズマは不安定になる。
この\textcolor{red}{ドリフト不安定}と呼ばれる現象については後述する。なお、grad B ドリフトも有限ラーモア半径効果であり、同様に電荷分離を引き起こす。しかし\eqref{df.grad_B_drift}式によれば、$v_{\nabla B}$は$kr_L$に比例するのに対し、\eqref{time_vary_general}式の補正項は$k^2r_L^2$に比例する。
したがって、grad B ドリフトは小さい$k$、すなわち比較的大きなスケール(長波長)で重要であるのに対し、非一様電場効果は、比較的大きな$k$、すなわち不均一性の小スケール長において重要となる。この理由から、ドリフト不安定性は\textcolor{red}{微小不安定性}と呼ばれるより一般的な分類に属する。

\subsection{時間変動する電場}
$\bm{E}$、$\bm{B}$が空間的には一様であるが、時間的に変化する場合を考える。まず初めに、$\bm{E}$のみが時間に関して正弦的に変化する場合を考える。x軸を$\bm{E}$に沿ってとり、

\begin{equation}
  \bm{E} = E_0 e^{i\omega t}\hat{\bm{x}}
\end{equation}
とする。$\dot{E}_x = i\omega E_x$から、\eqref{ddot_v_x}式は次のように書き換えられる。

\begin{equation}
  \ddot{v}_x =-\omega_c ^2\left(v_x \mp \frac{i\omega}{\omega_c}\frac{\tilde{E}_x}{B}\right)
\end{equation}
ここで、次のような量を定義する。

\begin{align}
  \begin{split}
    \tilde{v}_p &\equiv \pm \frac{i\omega}{\omega_c}\frac{\tilde{E}_x}{B}\\
    \tilde{v}_E &\equiv -\frac{\tilde{E}_x}{B}  
  \end{split}
\end{align}

上付きの$\, \tilde{}\, $は単にドリフトが振動していることを強調するためである。これにより、\eqref{ddot_v_x}、\eqref{ddot_v_x}式は次のように書き換えられる。

\begin{align}
  \begin{split}
    \ddot{v}_x = -\omega^2_c(v_x - \tilde{v}_p)\\
    \ddot{v}_y = -\omega^2_c(v_y - \tilde{v}_E)
  \end{split} \label{2.63}
\end{align}
この形は\eqref{base_eq}式と同じである。同様の推論から、\eqref{2.63}式の解を

\begin{align}
  \begin{split}
    v_x = v_\perp e^{i\omega_c t} + \tilde{v}_p\\
    v_y = \pm iv_\perp e^{i\omega_c t} + \tilde{v}_E
  \end{split} \label{2.64}
\end{align}
と仮定する。時間に対して2階微分すると、

\[
  \ddot{\tilde{v}}_p = \pm \frac{i\omega}{\omega_c}\frac{\ddot{\tilde{E}}_x}{B} = \pm \frac{i \omega}{\omega_c}\frac{1}{B}\cdot (-\omega^2)E_x = -\omega^2 \tilde{v}_p
\]
から、

\begin{align}
  \ddot{v}_x &= -\omega^2_c v_\perp e^{i\omega_c t} - \omega^2 \tilde{v}_p = -\omega^2_c(v_\perp e^{i\omega_c t}+\tilde{v}_p)+\omega^2_c \tilde{v}_p -\omega^2\tilde{v}_p \nonumber \\
             &= -\omega^2_c v_x + (\omega^2_c - \omega^2)\tilde{v}_p
\end{align}
$v_y$についても同様に式変形することで、

\begin{equation}
  \ddot{v}_y = -\omega^2_c v_y + (\omega^2_c - \omega^2)\tilde{v}_E
\end{equation}
\eqref{2.63}式と比較すると、$\omega^2\, \ll \, \omega^2_c$を満たすとき($\bm{E}$がゆっくり変化するとき)、仮定した解は正しかったことになる。

\eqref{2.64}式は案内中心が2つの成分を持っていることを表している。y成分は、$v_E$が周期$\omega$でゆっくりと振動していることを除けば、通常の$\bm{E}\times\bm{B}$ドリフトである。x成分は$\bm{E}$方向の新しいドリフトであり、\textcolor{red}{分極ドリフト}と呼ばれる。
$i\omega$を$\partial/\partial t$に変えることで、分極ドリフトは次のように表される。

\begin{equation}
  \boxed{\bm{v}_p = \pm\frac{1}{\omega_c B}\frac{d\bm{E}}{dt}} \label{2.66}
\end{equation}

この物理的説明は次のように理解できる。通常、電場と磁場が一定であれば、粒子は磁場に垂直な方向へ一定速度で$\bm{E}\times\bm{B}$ドリフト する。しかし、電場が時間とともに変化する場合、粒子はその変化にすぐには追随できない。電場が増加するとき、粒子は最初に電場の方向へ押されて加速されるが、磁場の中にあるため、
動き出した瞬間にローレンツ力を受けて軌道が曲がり、わずかに横方向へドリフトする。この横方向への一瞬の運動が分極ドリフトである。
電場が一定に保たれれば、粒子は最終的に定常的な$\bm{E}\times\bm{B}$ドリフト運動へと移行し、分極ドリフトは消える。しかし、電場が再び変化(増加・減少・反転)すると、その都度、粒子は慣性により再び$\bm{E}$の変化方向へ押され、短時間だけ余分なドリフトを生じる。
したがって、分極ドリフトは「電場の時間変化に対して粒子が慣性で遅れて反応するために生じる一時的ドリフト」と理解できる。電場の変化率$d\bm{E}/dt$が大きいほど慣性効果は大きく、磁場が強いほどローレンツ力が速く作用して慣性運動が抑えられるため、ドリフトは小さくなる。
電場の変化がゆるやか($\omega/\omega_c \ll 1$)であればほとんど無視できる。

$\bm{v}_p$はイオンと電子で異なる方向を向くため、プラズマ全体として一時的な分極が生じる。その分極が時間的に変化することによって、電流のように見える運動が現れる。この電流は分極電流と呼ばれる。Z=1の場合、

\begin{equation}
  \bm{j}_p = ne(\bm{v}_{ip} - \bm{v}_{ep}) = \frac{n}{B^2}(M+m)\frac{d\bm{E}}{dt} = \frac{\rho}{B^2}\frac{d\bm{E}}{dt}
\end{equation}
と表される。ここで、$\rho$は質量密度である。

プラズマ中の分極の効果は、固体誘電体における分極$\bm{D} = \bm{E} + 4\pi\bm{P}$と同じである。固体誘電体では、1つの原子の中で原子核と電子が、外部電場$\bm{E}$によってわずかに引き離されることで双極子ができる。
この微小な分離がたくさん集まり、マクロには分極ベクトル$\bm{P}$が現れる。プラズマでは、電子とイオンの相対的な位置のずれが双極子としてふるまう。つまり、プラズマの双極子は「ラーモア半径だけ分離した電子とイオンのペア」とみなすことができる。そのスケールは$r_L$程度となる。固体誘電体と異なる点は、プラズマではイオンと電子が準中性を保つように動き回るため、定常な電場であれば分極電場$\bm{P}$は生じない。
しかし、$\bm{E}$が振動していると、振動電流$\bm{j}_p$が慣性の遅れから発生する。


\subsection{時間変動する磁場}
最後に、磁場が時間的に変化する場合を考えよう。磁場が時間変化しない場合はローレンツ力は仕事をしないので荷電粒子にエネルギーを与えることはない。しかし、時間変化する場合、マクスウェル方程式から、

\begin{equation}
  \nabla \times \bm{E} = -\dot{\bm{B}}
\end{equation}
より、電場が生じ、これが粒子を加速(減速)させる。こうなるともはや、電磁場は一様であるとはいえない。磁場に垂直な面での軌道に沿った粒子の速度を$\bm{v}_\perp = d\bm{l}/dt$とする。ここで、$v_\parallel$の方向は考えない。
運動方程式\eqref{finite.E}式の$\bm{v}_\perp$方向について、$\bm{v}_\perp$と内積を取ると、

\begin{equation}
  \frac{d}{dt}\left(\frac{1}{2}mv^2_\perp\right) = q\bm{E}\cdot\bm{v}_\perp = q\bm{E}\cdot \frac{d\bm{l}}{dt}
\end{equation}
となり、1回転の運動エネルギーの変化量は、1周期における積分によって得られる。

\[
  \delta\left(\frac{1}{2}mv^2_\perp\right) = \int_{0}^{2\pi/\omega_c}q\bm{E}\cdot \frac{d\bm{l}}{dt}dt
\]
もし場が1周期の間にほとんど変わらないのであれば、軌道を「摂動されていない円軌道(半径$r_L$の円)」として扱える。よって粒子は閉じた軌道を回るので、時間積分は軌道に沿った線積分(周回積分)に置き換えられる。

\begin{align}
  \delta\left(\frac{1}{2}mv^2_\perp\right) = \oint_c q\bm{E}\cdot d\bm{l} = q\int_{S}(\nabla\times\bm{E})\cdot d\bm{S} &= -q\int_{S}\dot{\bm{B}}\cdot d\bm{S} \nonumber \\
                                                                                                                       &= \mp q\int_{S}\dot{\bm{B}}\cdot d\bm{S}
\end{align}
ここで、$\bm{S}$は$\bm{v}$に対して右ねじの方向を正とすると、プラズマは反磁性であるから、イオンで$\dot{\bm{B}}\cdot d\bm{S} < 0$ 、電子で$ <0$となることから、

\begin{equation}
  \delta\left(\frac{1}{2}mv^2_\perp\right) = \pm q\dot{B}\pi r_L ^2 = \pm q\pi \dot{B} \frac{v^2_\perp}{\omega_c}\frac{m}{\pm qB} = \frac{\frac{1}{2}mv^2_\perp}{B}\cdot \frac{2\pi \dot{B}}{\omega_c}
\end{equation}
さらに、$2\pi \dot{B}/\omega_c = \dot{B}/f_c = \dot{B}T$は1回転での$\delta B$の変化量であるから、

\begin{equation}
  \delta\left(\frac{1}{2}mv^2_\perp\right) = \mu \delta B
\end{equation}
となり、左辺は$\delta(\mu B)$であるから、

\[
  \delta(\mu B) =  \mu (\delta B) + (\delta \mu)B  =  \delta(\mu B) - (\delta \mu)B
\]
から、最終的に、

\begin{equation}
  \delta \mu = 0 \label{2.73}
\end{equation}
が得られる。これが示すことは、「ゆっくり変化する磁場中の磁気モーメントは不変である。」

磁場の強さが変化する場合では、ラーモア半径は広がったり、縮んだりし、粒子は磁場に垂直な方向のエネルギーを失ったり獲得したりする。この粒子と場との間のエネルギーの交換関係が\eqref{2.73}式によって簡単に表されている。$\mu$の不変性により、次のよく知られた定理:
\begin{center}
  Larmor 軌道に囲まれた磁束は一定である。
\end{center}
を簡単に証明することができる。

磁束$\Phi$は、$\Phi = BS$及び$S = \pi r^2_L$から、

\begin{equation}
  \Phi = B\pi \frac{v^2_\perp}{\omega^2_c} = B\pi \frac{v^2_\perp m^2}{q^2B^2} = \frac{2\pi m}{q^2}\frac{\frac{1}{2}mv^2_\perp}{B} = \frac{2\pi m}{q^2}\mu
\end{equation}
よって、$\mu$が一定であれば$\Phi$も一定になる。

この性質は、プラズマの加熱法の1つとして知られる断熱圧縮に用いられている。図\refeq{fig2.13}はこれがどのようにして行われるかを示した図である。まず、ミラーAとBの間の領域にプラズマを入射する。
コイルAとBの電流を増加させ、$\bm{B}$を増やす。すると、$\mu$は不変であるから$v^2_\perp$も増えることになる。コイルAの電流をさらに強めることでBの方が相対的に弱くなり、C-D側に押し出される形でプラズマを移動させることができる。
その上コイルCとDの電流を増やすことによって、プラズマをさらに圧縮加熱するわけである。 初期の磁気ミラー核融合装置はこの種の加熱方式を採用していた。断熱圧縮はトロイダルプラズマにおいても用いられており、磁気閉じ込めまたは慣性閉じ込めのいずれを用いるレーザー駆動核融合において不可欠な要素である。

\begin{figure}[H]
  \centering
  \includegraphics[width=0.7\linewidth]{fig2.13.png}
  \caption{プラズマの2段階断熱圧縮}
  \label{fig2.13}
\end{figure}


\subsection{案内中心のドリフトのまとめ}
これまで扱ったドリフトを以下にまとめる。

\begin{align}
\text{一般的な力 } &: \quad 
    \bm{v}_f = \frac{1}{q} \frac{\bm{F} \times \bm{B}}{B^2} \\
\text{電場} &: \quad 
    \bm{v}_E = \frac{\bm{E} \times \bm{B}}{B^2} \\
\text{重力場} &: \quad 
    \bm{v}_g = \frac{m}{q} \frac{\bm{g} \times \bm{B}}{B^2} \\
\text{非一様電場} \bm{E} &: \quad 
    \bm{v}_E = \left( 1 + \frac{1}{4}r_L^2\nabla^2 \right)\frac{\bm{E} \times \bm{B}}{B^2} \\
\end{align}
非一様磁場、電場

\begin{align}
\text{grad B ドリフト} &: \quad 
  \bm{v}_{\nabla B} = \pm \frac{1}{2} v_\perp r_L \frac{\bm{B} \times \nabla B}{B^2} \label{2.24}\\
\text{湾曲ドリフト} &: \quad 
  \bm{v}_R = \frac{m v_\parallel^2}{q}\frac{\bm{R}_c \times \bm{B}}{R_c^2 B^2} \label{2.26}\\
\text{湾曲真空磁場} &: \quad 
  \bm{v}_R + \bm{v}_{\nabla B} = \frac{m}{q} \left( v_\parallel^2 + \frac{1}{2}v_\perp^2 \right)\frac{\bm{R}_c \times \bm{B}}{R_c^2 B^2}\\
\text{分極ドリフト} &: \quad 
  \bm{v}_p = \pm \frac{1}{\omega_c B} \frac{d\bm{E}}{dt}
\end{align}


\subsection{断熱不変量}
古典力学でよく知られているように、ある系が周期的運動する場合に、一回りにわたっての作用積分$\oint pdq$は常に一定となる。ここで、p,qは一般運動量と一般座標である。系の変化がゆっくりとしている場合には厳密には周期運動とはいえないが、作用積分は変化しない。したがって、この恒量は断熱不変量と呼ばれる。

\begin{eqbox}{断熱不変量}
\begin{equation}
  J = \oint pdq
\end{equation}
\text{という作用積分は、パラメータがゆっくり変化する系において不変である。}
\end{eqbox}
ここでいう、ゆっくりというのは、運動の周期に比べてゆっくりということであり、作用積分$\oint pdq$は厳密には閉じた経路にわたる積分でないが、定義される断熱不変量は、プラズマ物理において重要な役割を果たす。
つまり、これを用いることにより、複雑な運動を含んだ様々な場合について簡単な答えを得ることができる。異なったタイプの周期運動に対応して、3つの断熱不変量が存在する。

\subsubsection{第1断熱不変量, $\mu$}
すでに
\[
  \mu = mv^2_\perp / 2B
\]
という量を導入し、空間的および時間的に変化する磁場においてその不変性を証明した。その周回運動はもちろんラーモア回転である。$p$を$mv_\perp r$、$dq$を$d\theta$とすると、作用積分は、

\begin{equation}
  \oint pdq = \oint mv_\perp r_L d\theta = 2\pi r_L mv_\perp = 2\pi \frac{mv^2_\perp}{\omega_c} = 4\pi \frac{m}{|q|}\mu
\end{equation}
したがって、$\mu$は断熱不変量といえる。$\mu$の不変性は$\omega/\omega_c \ll 1$を仮定して証明した。しかし、$\omega \leq \omega_c$のときでも$\mu$の不変性を証明することができる。回転の1周期の間における$\bm{B}$の変化に比べて$\mu$はほとんど定数であることを意味する。

断熱不変量が存在しない場合を知ることも重要である。$\omega$が$\omega_c$に比べて小さくない場合、$\mu$の断熱不変性は破られる。これに関する三つの例を示す。

(A) 磁気ポンプ\\
磁気ミラー装置の中で、磁場の強さを正弦的に変化させると、粒子の$v_\perp$が振動することになるが、エネルギーの利得はない。しかし、粒子が衝突を起こすと$\mu$の不変性が壊られる。圧縮段階で衝突した粒子は、回転運動のエネルギー($v_\perp$)の一部を並進運動($v_\parallel$)に変換する。
このエネルギーは膨張段階で戻らないため、プラズマが加熱される。

(B) サイクロトロン加熱\\
磁場を$\omega_c$で振動させると、誘起された電場も同じ周波数で回転する。この電場がいくつかの粒子と一緒にまわり、粒子の運動エネルギーを継続的に増やす。このとき$\omega \approx \omega_c$であり、条件$\omega \ll \omega_c$が破れるため、$\mu$は保存されず、プラズマは加熱される。

(C) 磁気カスプ\\
磁気カスプとは、磁場線(磁力線)が空間の一点から放射状に広がる構造のことを指す。磁気ミラー装置のコイルのうち1つの電流を逆にすると、中心に磁気カスプができる(図\refeq{fig2.14})。
カスプ装置に閉じ込められたプラズマは、普通のミラー装置に比べて良い安定性を持つ。しかし、ロスコーン損失ははるかに大きくなる。その理由の1つとして、装置の中心(カスプ点)では磁場が0であり、$\mu$は無限大に発散するかのように見え、保存されない。
また、カスプ点では、磁場が急激に弱まるため、$r_L$が急に大きくなり、ロスコーンの外側に存在した粒子がロスコーンの中に入ってくるかもしれない。結果として、プラズマ閉じ込め性能が低下する。
幸いなことに、この場合には、別の不変量が存在する;正準角運動量 $p_\theta = mrv_\theta - erA_\theta$である。これは、衝突を起こすまで粒子集団が無期限に閉じ込められることを保証している。

\begin{figure}[H]
  \centering
  \includegraphics[width=0.6\linewidth]{fig2.14.png}
  \caption{磁気カスプ中のプラズマ閉じ込め}
  \label{fig2.14}
\end{figure}


\subsubsection{第2断熱不変量, $J$}
2つの磁気ミラーの間に捕捉された粒子を考える。粒子はミラー間を往復し、その『往復周波数』で周期運動する。$ds$を磁力線に沿っての案内中心の軌道の線素として、$\oint mv_\parallel ds$を考える。
案内中心は磁力線を横切ってドリフトするため、正確には周期運動ではないが、実はこの量が断熱不変量となる。これは縦の不変量と呼ばれ、2つの反射点の間の半周期の積分で定義される(図\refeq{fig2.14}):

\begin{equation}
  \boxed{J = \int_{a}^{b}v_\parallel ds} \label{2.76}
\end{equation}
厳密には周期的ではないこと、さらに1回りにわたる積分ではないことに注意してほしい。

\begin{figure}[htbp]
  \centering
  \includegraphics[width=0.6\linewidth]{fig2.15.png}
  \caption{ミラー磁場中をa-b間を往復する粒子}
  \label{fig2.15}
\end{figure}

$J$が断熱不変量であることを証明する前に、$J$が断熱不変量であると言えた場合のメリットを紹介しよう。すでに述べたように、地磁気はミラーを形成しており荷電粒子を捕捉するが、grad B ドリフトなどにより、その荷電粒子は地球の周りをゆっくりとドリフトする。
もし磁場が完全に対称であれば、粒子は同じ磁力線に戻ってくるが、実際には、太陽風などのためにひずみがあるので、このような場合には粒子は同一の磁力線に戻ってくるか問題である。
つまり、ドリフトで粒子が元の位置に戻ってきたときには、高度の異なる磁力線にいるかもしれない。もし$J$が保存されるのなら、このようなことは起こりえない。$J$は反射点の間の磁力線の長さを決めるため、$J$が保存するのであれば粒子は違う高さの磁力線へ行くことはないのである。
結果として、粒子は微妙に磁場が変化していても、同じ磁力線に戻ってくる。

\begin{figure}[htbp]
  \centering
  \includegraphics[width=0.7\linewidth]{fig2.16.png}
  \caption{地球磁場中の荷電粒子の運動}
  \label{fig2.16}
\end{figure}

$J$の不変性を証明するために、まず$v_\parallel\delta s$の不変性を証明する。ここで、$\delta s$は$\bm{B}$に沿った微小区間である(図\refeq{fig2.17})。
ドリフトにより、s上の粒子は$\Delta t$後に他の磁力線$\delta s'$上に存在する。図\refeq{fig2.17}より、それぞれの長さの関係は、

\[
  \delta s = R_c \delta \theta \hspace{1pc},\hspace{1pc} \delta s' = R'_c \delta \theta
\]
から、
\[
  \frac{\delta s}{R_c} = \frac{\delta s'}{R'_c}
\]
となる。この関係を用いて、

\[
  \delta s' - \delta s = \frac{R'_c}{R_c}\delta s - \delta s = \delta s \frac{R'_c - R_c}{R_c}
\]
両辺を$\Delta t$で割ることにより

\begin{equation}
  \frac{\delta s' - \delta s}{\Delta t \delta s} = \frac{R'_c - R_c}{\Delta t R_c} \label{2.77}
\end{equation}
を得る。曲率半径方向の変化を考えると、

\begin{equation}
  R'_c - R_c = \bm{e}_r \cdot \Delta r \label{pre2.78}
\end{equation}
ここで、$\bm{e}_r$は曲率半径方向の単位ベクトルで、$\bm{R_c}/R_c$と表せる。また、$\Delta r$は案内中心が進んだ距離で、$\bm{v}_{gc}\Delta t$と表せる。したがって、\eqref{pre2.78}式は、

\[
  R'_c - R_c = \frac{\bm{R}_c}{R_c}\cdot \bm{v}_{gc}\Delta t
\]
$\Delta t$で割ることで、

\begin{equation}
  \bm{v}_{gc} \cdot \frac{\bm{R}_c}{R_c} = \frac{R'_c - R_c}{\Delta t} \label{2.78}
\end{equation}

\begin{figure}[htbp]
  \centering
  \includegraphics[width=0.6\linewidth]{fig2.17.png}
  \caption{$J$の不変性の証明}
  \label{fig2.17}
\end{figure}

非一様磁場でのドリフトはgrad B ドリフトと湾曲ドリフトで表せるのであった。\eqref{2.24}、\eqref{2.26}式より、

\begin{equation}
  \bm{v}_{gc} = \bm{v}_{\nabla B} + \bm{v}_R = \pm \frac{1}{2}v_\perp r_L \frac{\bm{B}\times \nabla B}{B^2} + \frac{mv^2_\parallel}{q}\frac{\bm{R}_c \times \bm{B}}{R^2_cB^2} \label{2.79}
\end{equation}
\eqref{2.78}、\eqref{2.79}式を用いて\eqref{2.77}式を書き直すと、

\[
  \bm{v}_{gc}\cdot \frac{\bm{R}_c}{R^2_c} = \frac{\delta s' - \delta s}{\Delta t \delta s}
\]
さらに右辺を書き換えることで、

\begin{equation}
  \frac{1}{\delta s}\frac{d}{dt}\delta s = \bm{v}_{gc}\cdot \frac{\bm{R}_c}{R^2_c} = \frac{1}{2}\frac{m}{q}\frac{v^2_\parallel}{B^3}(\bm{B}\times \nabla B)\cdot \frac{\bm{R}_c}{R^2_c} \label{2.80}
\end{equation}
ここで、$\bm{v}_R$の項は内積をとったときに落ちていることに注意してほしい。

$J$の不変性の証明には$v_\parallel$を求めないといけない。水平・垂直両成分のエネルギーは、

\begin{equation}
  W = \frac{1}{2}mv^2_\parallel + \frac{1}{2}mv^2_\perp = \frac{1}{2}mv^2_\parallel + \mu B \equiv W_\parallel + W_\perp \label{2.81}
\end{equation}
これから、

\begin{equation}
  v_\parallel = \left[\frac{2(W-\mu B)}{m}\right]^{1/2} \label{2.82}
\end{equation}
と書ける。ここで、$W$と$\mu$は定数なので、変数は$B$のみである。
\[
  \dot{v}_\parallel = \left(\frac{2}{m}\right)^{1/2}\cdot \frac{1}{2}(W-\mu B)^{-1/2}\cdot (-\mu \dot{B}) = -\frac{\mu B}{2}\left[\frac{2}{m(W-\mu B)}\right]^{1/2}
\]
から、

\begin{equation}
  \frac{\dot{v}_\parallel}{v_\parallel} = -\frac{1}{2}\frac{\mu \dot{B}}{W-\mu B} = \frac{\mu \dot{B}}{mv^2_\parallel} \label{2.83}
\end{equation}
また、$\dot{\bm{B}}$を書きなおすと、

\begin{equation}
  \dot{\bm{B}} = \frac{dB}{d\bm{r}}\cdot \frac{d\bm{r}}{dt} = \nabla B\cdot \bm{v}_{gc} = \frac{mv^2_\parallel}{q}\frac{\bm{R}_c\times \bm{B}}{R^2_cB^2}\cdot \nabla B \label{2.84}
\end{equation}
ここでは$\bm{v}_{\nabla B}$の項を内積をとったときに落としている。これを用いて\eqref{2.83}式を書きなおすと、

\begin{equation}
  \frac{\dot{v}_\parallel}{v_\parallel} = -\frac{\mu}{q}\frac{(\bm{R}_c \times \bm{B})\cdot \nabla B}{R^2_cB^2} = -\frac{1}{2}\frac{m}{q}\frac{v^2_\perp}{B}\frac{(\bm{B}\times \nabla B)\cdot \bm{R}_c}{R^2_cB^2} \label{2.85}
\end{equation}
ここで、$(\bm{A}\times \bm{B})\cdot \bm{C}$の形はサイクリックに順番を変えるのであれば等しくなるという性質を用いている。

$v_\parallel \delta s$の変化率は

\begin{equation}
  \frac{1}{v_\parallel \delta s}\frac{d}{dt}(v_\parallel \delta s) = \frac{1}{\delta s}\frac{d\delta s}{dt} + \frac{1}{v_\parallel}\frac{dv_\parallel}{dt} \label{2.86}
\end{equation}
\eqref{2.80}、\eqref{2.85}式から、2つの項は打ち消し合い、

\begin{equation}
  v_\parallel \delta s = \text{constant} \label{2.87}
\end{equation}
が得られる。

$J$の不変性が壊られる例として、 遷移時間磁気ポンプと呼ばれるプラズマ加熱法を挙げる。その装置では磁気ミラーのコイルの電流を時間的に変化させる。すると、磁場の大小の変化に伴いミラー位置が周期的に近づいたり、離れたりする。
図\refeq{fig.mirror}において、磁場が強くなると元のすぼめ口が左にシフトし、もともとの位置にはよりすぼんだ口ができることをイメージすればいい。粒子の往復振動数に近い周波数で振動電流を加えると、ちょうどその往復周波数を持つ粒子は、ミラーに向かって進むたびにミラーの方からも近づいてくるように見える。
その結果、粒子は毎回$v_\parallel$が増加することになる。これも動いてくる壁にボールをぶつけると、ボールは速度を増して跳ね返ってくることをイメージすればよい。この場合、磁場が粒子の往復運動に対して十分ゆっくりではないため、$J$は保存されない。


\subsubsection{第3断熱不変量, $\Phi$}
図\refeq{fig2.16}を再び参照すると、地球まわりを動く案内中心のゆっくりとしたドリフトが第3の周期運動を構成することがわかる。これに対応する断熱不変量は、ドリフト表面により囲まれた磁束$\Phi$の総和である。磁場の変化が「十分ゆっくり」である場合、系は連続的に準静的に変形し、粒子は囲んだ磁力線の総和が一定であるような面上にとどまるように動く。
しかし、この不変量$\Phi$はあまり応用例がない。というのも、太陽風やMHD波などにより引き起こされる$\bm{B}$の時間変動はドリフトの周期に比べて短い時間尺度で起こることが多い。そのため、$\Phi$は保存されない。



\section{流体としてのプラズマ}
\subsection{導入}
プラズマでは、前章の状況よりもはるかに複雑である。電場と磁場は与えられたものではなく、電荷自体の位置と運動によって決定される。すなわち、粒子が軌道に沿って移動する際に場を生成し、かつ場が粒子をそれらの正確な軌道上を移動させるようにする。このような自己整合的な問題を解かなければならない。さらにこれは時間変動する状況下で行わなければならない。

典型的なプラズマ密度は1$\si{\metre^2}$あたり$10^{18}$個のイオン・電子対である。これらの粒子がそれぞれ複雑な軌跡を描き、それらすべてを追跡する必要があるならば、プラズマの挙動を予測することは絶望的である。幸いなことに、実際の実験で観測されるプラズマ現象の大部分(おそらく80\%ほど)は、かなり大雑把なモデルで説明できる。
このモデルは流体力学で用いられるもので、個々の粒子の同一性は無視され、流体要素の動きのみが考慮される。衝突が一般的に稀なプラズマにこのようなモデルが適用できるのは驚くべきことだと思うが、これの理由があることが後でわかるだろう。
この章ではプラズマの流体理論から得られる事実を述べる。より洗練されたプラズマの運動論は7章で扱う。

一部のプラズマ問題では、流体理論も運動論もプラズマの挙動を記述するには不十分である。その場合、個々の軌跡を追跡するという煩雑なプロセスをコンピュータによって計算するしかない。理論と実験の間の隔たりを埋める上で、コンピュータシミュレーションは重要な役割を果たしている。

\subsection{運動の流体方程式}
\subsubsection{対流微分}

粒子1個の運動方程式は、
\begin{equation}
  m\frac{d\bm{v}}{dt} = q(\bm{E}+\bm{v}\times \bm{B}) \label{normal.eq}
\end{equation}
衝突も熱運動もないと仮定すると流体要素内の全粒子は共に運動し、要素内の粒子の平均速度$\bm{u}$は個々の粒子速度$\bm{v}$と同一となる。流体方程式は\eqref{normal.eq}式に密度nを乗じるだけで得られ、

\begin{equation}
  mn\frac{d\bm{u}}{dt} = qn(\bm{E}+\bm{u}\times \bm{B}) \label{fluid_eq}
\end{equation}
となる。しかし、この方程式は使うのには不便である。というのも、\eqref{fluid_eq}式は粒子とともに移動する座標系で成り立っている。我々が欲しいのは空間に固定された座標での流体方程式である。
ある点$\bm{G}$におけるこれらの座標系の変換は、

\begin{equation}
  \frac{d\bm{G}}{dt} = \frac{\partial \bm{G}}{\partial t} + (\bm{u}\cdot \nabla)\bm{G}
\end{equation}
で表される。これの導出は\url{https://eman-physics.net/fluid/lagrange_derivative.html}を見てほしい。この関係式は対流微分もしくはラグランジュ微分と呼ばれる。
左辺は「流体要素(粒子)と一緒に動く立場」で見たときの時間変化であり、右辺第1項は「固定した位置」での時間の変化である。右辺第2項は移動による変化の寄与であり、対流項と呼ばれる。
次に示すのは、この対流微分を直感的に理解する例である。図\refeq{fig3.1}のような電気ケトルの中では、温められた水は上に上がり、冷たい水は下に沈む。左辺:流体要素(粒子)と一緒に動く立場は、もちろん流体要素の温度変化を表す。
右辺第2項:移動による変化の寄与は、対流で生じた熱の流れが加わることを意味している。つまり、
\begin{center}
  (流体要素の温度変化) = (その場での温度変化) + (流れてくる温度)
\end{center}
を表している。

\begin{figure}[H]
  \centering
  \includegraphics[width=0.4\linewidth]{fig3.1.png}
  \caption{電子ケトル中での流体要素の運動}
  \label{fig3.1}
\end{figure}


\subsubsection{応用テンソル}
単一の粒子を考えていたときと異なる点は、流体要素の中や外の粒子のランダムな運動によって生じる圧力項が加わることである。流体要素$\Delta x \Delta y \Delta z$を考える(図\eqref{fig3.3})。簡単のため、面AとBとを通る運動のx成分のみを考える。
速度が$v_x$である粒子数は、固定した$v_x$に対して、
\[
  \iint f(v_x, v_y, v_z)dv_y dv_z
\]
完全に$v_x$を持つ粒子は存在しないので、粒子数を数えるには幅 $\Delta v_x$を考える必要がある。したがって、速度$v_x$を持つ粒子の単位体積あたりの数は
\[
  \Delta n_v = \Delta v_x \iint f(v_x, v_y, v_z)dv_y dv_z
\]
となる。

\begin{figure}[htbp]
  \centering
  \includegraphics[width=0.7\linewidth]{fig3.3.png}
  \caption{応用テンソルの各成分}
  \label{fig3.3}
\end{figure}

面Aを通って、流体要素の中に流れ込んでくるx方向の運動量は、(面Aを通過する粒子の数)$\times$ (各粒子が持つ運動量$mv_x$)で求められる。
面Aを通る粒子数は前式で求めた$\Delta n_v$を用いて、
\[
  \Delta n_v v_x \Delta y \Delta z
\]
と表せるから、速度$v_x$を持つ粒子の流れ込んでくるx方向の運動量は
\[
  \Delta n_v mv^2_x \Delta y \Delta z
\]
となる。求めたいのは、面Aを通って、流体要素の中に流れ込んでくるすべての粒子のx方向の運動量である。そのためには、粒子が持つ速度のバラエティーをすべて足し合わせればよい。
よって、$x_0$においてAを通して要素内に運ばれる運動量$P_{A+}$は
\begin{equation}
  P_A = \sum \Delta n_v mv^2_x \Delta y \Delta z = \Delta y \Delta z\left[m\overline{v^2_x}\frac{1}{2}n\right]_{x_0 - \Delta x} 
\end{equation}
ここで最後の変形には、平均速度の2乗が
\[
  \overline{v^2_x} = \frac{1}{n}\sim \Delta n_v v^2_x
\]
と表せることを用いた。また、ちょうど面Aに流れ込む粒子は$x_0- \Delta x$の位置にいた粒子であり、そのうち半分が右方向に進むことを用いている。
同様に、面Bを通って要素内から出ていく運動量は

\[
  P_B = \Delta y \Delta z \left[m\bar{v^2_x}\frac{1}{2}n\right]_{x_0}
\]
よって、要素内が得る正味の運動量は

\begin{equation}
  P_A - P_B = \Delta y \Delta z\frac{1}{2}m\left([n\bar{v^2_x}]_{x_0 - \Delta x} - [n\bar{v^2_x}]_{x_0}\right) = \Delta y \Delta z \frac{1}{2}m (-\Delta x)\frac{\partial}{\partial x}(n\bar{v^2_x}) \label{3.37}
\end{equation}

巨視的な流体の速度を$u_x$とすると、ここまでの議論から$x_0$にある流体要素の運動量の時間変化は

\begin{equation}
  \frac{\partial}{\partial t}(nmu_x)\Delta x \Delta y \Delta z = -m\frac{\partial}{\partial x}(n\overline{v^2_x})\Delta x \Delta y \Delta z \label{3.38}
\end{equation}

$v_x$を次の2つに分けて考える。
\[
 v_x = u_x + v_{xr} \hspace{2pc} u_x = \bar{v}_x
\]
2つ目の仮定から、$v_{xr}$は粒子のランダムな熱速度を表す。このランダムな熱速度がMaxwell分布に従うとすると、1次元では

\begin{equation}
  \frac{1}{2}m\overline{v^2_{xr}} = \frac{1}{2}KT \label{3.39}
\end{equation}
であったことから、\eqref{3.38}式は

\[
  \frac{\partial}{\partial t}(nmu_x) = -m\frac{\partial}{\partial x}\left[n(\overline{u^2_x} + 2\overline{uv_{xr}} + \overline{v^2_{xr}})\right] = -m\frac{\partial}{\partial x}\left[n\left(u^2_x + \frac{KT}{m}\right)\right]
\]
となる。最後の式変形には$\overline{v_{xr}} = 0$を用いた。両辺微分を進めると

\begin{equation}
  mn\frac{\partial u_x}{\partial t} + mu_x\frac{\partial n}{\partial t} = -mu_x\frac{\partial (nu_x)}{\partial x} - mnu_x\frac{\partial u_x}{\partial x} - \frac{\partial}{\partial x}(nKT) \label{3.40}
\end{equation}
これに連続の式(\ref{section3.3.5}節で扱う):

\begin{equation}
  \frac{\partial n}{\partial t} + \frac{\partial}{\partial x}(nu_x) = 0 \label{3.41}
\end{equation}
を加えることで、2つの項が消去され
\[
  mn\frac{\partial u_x}{\partial t} = - mnu_x\frac{\partial u_x}{\partial x} - \frac{\partial}{\partial x}(nKT) 
\]
となる。圧力を
\begin{equation}
  \boxed{p \equiv nKT} \label{3.42}
\end{equation}
と定義することで、最終的に

\begin{equation}
  mn\left(\frac{\partial u_x}{\partial t} + u_x \frac{\partial u_x}{\partial x}\right) = -\frac{\partial p}{\partial x} \label{3.43}
\end{equation}
が得られる。これが通常の圧力勾配による力である。電磁力を加え、3次元へ拡張することで次の流体方程式が得られる。

\begin{equation}
  mn\left[\frac{\partial \bm{u}}{\partial t} + (\bm{u}\cdot \nabla)\bm{u}\right] = qn(\bm{E} + \bm{u}\times \bm{B}) - \nabla p \label{3.44}
\end{equation}

ここに導いた式はx方向の運動量の輸送がx方向の運動のみ起因する(x,y方向も同様)といった特殊な場合にすぎない。もっと一般的には、y方向の運動によりx方向の運動量の輸送が起こるなど他の方向の寄与も考えないといけない。
このずれ応力はスカラー$p$で表すことはできず、テンソル$\bm{P}$で表さなければならない。これを\textcolor{red}{応用テンソル}と呼ぶ。$-\nabla p$の項は$-\nabla \cdot \bm{P}$で置き換えられる。

2つの極めて簡単な場合の応用テンソルを紹介しよう。分布関数が等方的なMaxwell型であるときには、$\bm{P}$は次のようになる。

\begin{equation}
  \begin{pmatrix}
    p & 0 & 0\\
    0 & p & 0\\
    0 & 0 & p
  \end{pmatrix} \label{3.45}
\end{equation}
この場合、$\nabla \cdot \bm{P}$は$\nabla p$と等しくなる。\ref{section1.3}節で、磁場があるときにはプラズマは2つの温度$T_\perp$と$T_\parallel$を持ちうることを述べた。この場合、2つの圧力$p_\perp = nk_BT_\perp$と$p_\parallel = nk_BT_\parallel$が存在し、
応用テンソルは

\begin{equation}
  \begin{pmatrix}
    p_\perp & 0 & 0\\
    0 & p_\perp & 0\\
    0 & 0 & p_\parallel
  \end{pmatrix} \label{3.46}
\end{equation}
となる。3行目あるいは3列目の座標は$\bm{B}$の方向を指す。なお$\bm{B}$に垂直な平面では等方であることを示す。

通常の流体では、$\bm{P}$の非対角成分は粘性に関連する量を表す。粒子が衝突のたびにその場の流体速度を運び去るため、速度が場所によって均され、速度の違いに対する抵抗が生じる。これが我々が通常理解している粘性である。
平均自由行程が長ければ長いほど、粒子は衝突までにより遠くへ運動量を運ぶため、速度を均す作用が強まり、粘性係数は大きくなる。直感的には粘性が大きければ平均自由行程が短くなると感じるかもしれないが、日常で粘性が大きい流体(液体)は分子間力や構造の拘束によるものが多い。
粘性の本質は速度勾配に対して運動量がどれだけ輸送されるかである。一方、プラズマでは衝突がほとんどなくても、ラーモア運動によって粒子が異なる位置へ運動量を運ぶため、同様の「粘性のような効果」が生じる。この場合にスケールを決めるのは平均自由行程ではなくラーモア半径である。
これは有限ラーモア半径効果と呼ばれ、不均一な電場中でのドリフト$v_E$とも密接に関連している。


\subsubsection{衝突} \label{section3.3.3}
中性気体がある場合には、荷電粒子はそれらと衝突を通して運動量を交換する。1回の衝突で失われる運動量は、$\bm{u}_0$を中性流体の速度として、相対速度$\bm{u}-\bm{u}_0$に比例する。
もし衝突と衝突の間の平均自由時間$\tau$が近似的に定数であれば、生じる力の項は$-mn(\bm{u}- \bm{u}=0)/\tau$と書ける。この項も含めると、運動方程式\eqref{3.44}は次のように一般化される。

\begin{equation}
  mn\left[\frac{\partial \bm{u}}{\partial t} + (\bm{u}\cdot \nabla)\bm{u}\right] = qn(\bm{E} + \bm{u}\times \bm{B}) - \nabla \cdot \bm{P} - \frac{mn(\bm{u}-\bm{u}_0)}{\tau} \label{3.47}
\end{equation}
この方程式では荷電粒子間の衝突は含めていない。それは\ref{section5}章で扱う。


\subsubsection{流体力学との比較} \label{section3.3.4}
通常の流体はナビエ・ストークス方程式:

\begin{equation}
  \rho \left[ \frac{\partial \bm{u}}{\partial t} + (\bm{u}\cdot \nabla)\bm{u}\right] = -\nabla p + \rho \nu \nabla^2 \bm{u} \label{3.48}
\end{equation}
に従う。電磁力や何種類かの粒子間の衝突を除けば、これはプラズマの式\eqref{3.47}と同じである。$\nu$は動粘性係数であり、この粘性項$\rho \nu \nabla^2 \bm{u}$は$\nabla \cdot \bm{P}$に該当する。では、\eqref{3.47}式は本当にプラズマを表すといってもいいのだろうか。答えはYesであり、この理由が流体理論の限界を教えてくれる。

\eqref{3.48}式は粒子間の衝突の多い流体を記述するが、\eqref{3.47}式は特に衝突率を気にすることなく導いたように思われる。しかし、導いた過程を振り返ってみると、速度分布をMaxwell型としていた。この時点で衝突を暗に仮定していることになる。
この仮定は$v^2_{xr}$の平均をとるときのみ使われただけであるので、もし他の分布でも同じ平均値をとれば同じ答えになってしまう。つまり、流体理論は、Maxwell分布からのずれを無視してしまう可能性がある。
そのずれが重要となる例については\ref{section7}章で扱う。そこでは流体理論に変わって運動論が用いられる。

先に述べたように、大部分のプラズマは流体理論で扱うことができる。それを裏付ける実験として、Langmuirの実験がある。Langmuirは、衝突が少なくても電子速度分布がマクスウェルに近いことを発見した。このLangmuirのパラドックスと呼ばれる現象は、しばしば高周波振動のためとされているが、明確な理由は未解決のままである。
この現象は、プラズマ物理における「自然が味方している」数少ない例である。
流体モデルがプラズマに当てはまるもう一つの例としては、磁場がある意味で衝突の役割を果たすということである。例えば、電場$\bm{E}$によって粒子が加速される場合、本来なら速度は無限に増加するが、衝突が多ければ粒子の速度は$\bm{E}$に比例したある限界値に達する。
例えば、銅線中の電子は速度$\bm{v} = \mu \bm{E}$でドリフトする。$\mu$は易動度である。これが通常の衝突の効果である。
プラズマの場合には、磁場が、粒子が自由に動くことを制限し、Larmor軌道に沿って回転させる性質がある。プラズマ中の電子も、$\bm{E}$に比例した速度$\bm{v}_E = \bm{E}\times \bm{B}/B^2$でドリフトする。このような意味で、無衝突プラズマも衝突のある流体のようにふるまうといえる。
ただし注意すべき点は、これは主に磁場に垂直な方向の運動についての話であり、磁場に平行な方向への運動については、粒子は自由に飛んでしまうため流体理論は適用しにくい。逆に言えば、磁場に垂直な方向の物理に関しては流体理論が良い近似となる。


\subsubsection{連続の方程式} \label{section3.3.5}
物質の保存により、体積$V$中の粒子の総数$N$の変化は、$V$を囲む表面$S$から出ていく粒子数と等しいから、

\begin{equation}
  \frac{\partial N}{\partial t} = \int_{V}\frac{\partial n}{\partial t}dV = -\int_{S} n\bm{u}\cdot d\bm{S} = -\int_{V}\nabla \cdot (n\bm{u})dV \label{3.49} 
\end{equation}
最後の変形にはガウスの法則を用いた。これより、粒子数に対する連続の方程式:

\begin{equation}
  \frac{\partial n}{\partial t} + \nabla \cdot (n\bm{u}) = 0 \label{3.50}
\end{equation}
が得られる。


\subsubsection{状態方程式} \label{section3.3.6}
プラズマの流体方程式での未知量は、密度$n$、流速$\bm{u}$、圧力$p$の3つに対し、ここまでに得た方程式は、流体方程式と連続の方程式の2つ。よって、方程式群をひとまとめにするには、もう一つの関係式が必要となる。
それは$p$と$n$とを関係づける熱力学の状態方程式(ポアソンの法則):
\begin{equation}
  pV^\gamma = C\hspace{1pc}, \hspace{1pc} \frac{n'RT}{v}V^\gamma = C\hspace{1pc} \to \hspace{1pc} p = Cn^\gamma \label{3.51}
\end{equation}
である。ここで、$C$は定数、$\gamma$は比熱比$C_p/C_v$である。欲しいのは$\nabla p$であるから、両対数をとり、微分することで

\begin{equation}
  \frac{\nabla p}{p} = \gamma \frac{\nabla n}{n} \label{3.52}
\end{equation}
が得られる。等温圧縮では、
\[
  \nabla p = \nabla (nKT) = KT\nabla n
\]
であるため、$\gamma = 1$となる。断熱圧縮では、$KT$が変化し、$\gamma$は1より大きくなる。$N$を自由度とすれば、$\gamma$は次式で与えられる。

\begin{equation}
  \gamma = (2+N)/N \label{3.53}
\end{equation}

状態方程式が適用できるには、熱流が無視できること、すなわち熱伝導率が小さいことが求められる。これもまた、$\bm{B}$に平行な方向よりも$\bm{B}$に垂直な方向で成り立ちやすい。
幸いなことに、ほとんどの基礎的な現象は\eqref{3.51}式のような粗い近似によって十分に記述できる。


\subsubsection{流体方程式の完全なセット} \label{section3.3.7}
簡単のために、プラズマはイオンと電子の2種の粒子のみからなるとする。多種類の場合への拡張はそれほど意味はない。電荷密度と電流密度は次式で与えられる。

\begin{align}
  \begin{split}
    \sigma = n_iq_i + n_eq_e\\
    \bm{j} = n_iq_i\bm{v}_i + n_eq_e\bm{v}_e
  \end{split} \label{3.54}
\end{align}
もはや単一粒子の運動は考えないので、流体速度として$\bm{v}$の代わりに$\bm{u}$を用いる。衝突と粘性は無視する。これを踏まえて、Maxwell方程式および\eqref{3.44},\eqref{3.50},\eqref{3.51}式は次のようになる。

\begin{equation}
  \epsilon_0\nabla \cdot \bm{E} = n_iq_i + n_eq_e \label{3.55}
\end{equation}
\begin{equation}
  \nabla \times \bm{E} = -\dot{\bm{B}} \label{3.56}
\end{equation}
\begin{equation}
  \nabla \cdot \bm{B} = 0 \label{3.57}
\end{equation}
\begin{equation}
  \mu^{-1}_0 \nabla \times \bm{B} = n_iq_i\bm{v}_i + n_eq_e\bm{v}_e + \epsilon_0\dot{\bm{E}} \label{3.58}
\end{equation}
\begin{equation}
  m_jn_j\left[\frac{\partial \bm{v}_j}{\partial t} + (\bm{v}_j \cdot \nabla)\bm{v}_j\right] = q_jn_j(\bm{E} + \bm{v}_j \times \bm{B}) - \nabla p_j \hspace{2pc} j=i,e \label{3.59}
\end{equation}
\begin{equation}
  \frac{\partial n_j}{\partial t} + \nabla \cdot (n_j\bm{v}_j) = 0 \hspace{2pc} j=i,e \label{3.60}
\end{equation}
\begin{equation}
  p_j = C_jn^{\gamma_j}_j \hspace{2pc} j=i,e \label{3.61} 
\end{equation}
これらの方程式を同時に解くことで、流体近似における自己整合的な場と運動の体系が得られる。


\subsection{\texorpdfstring{$\bm{B}$}{B}に垂直な流体ドリフト} \label{section3.4}
流体要素は持つが、粒子は持たないドリフトを考える。$\nabla p$項は流体方程式にのみ現れるため、これに関連するドリフトが存在する。運動方程式は、各粒子群につき

\begin{equation}
  mn\left[\underline{\frac{\partial \bm{v}}{\partial t}}_{\text{\large ①}} + \underline{(\bm{v}\cdot\nabla)\bm{v}}_{\text{\large ②}}\right] = qn\left(\bm{E} + \underline{\bm{v}\times\bm{B}}_{\text{\large ③}}\right) - \nabla p \label{3.62}
\end{equation}
となる。ここでは、$\bm{v}_\perp$のみを考える。➂項はLarmor回転運動を表すため、この項に比べると、➀項($v_\perp$の時間変化)と➁項($v_\perp$の二次の項)ははるかに小さく、無視できる。したがって、\eqref{3.62}式は
\[
  0 = qn(\bm{E} + \bm{v}\times \bm{B}) - \nabla p
\]
としてもよい。これと$\bm{B}$との外積をとると
\begin{align*}
  0 &= qn[\bm{E}\times \bm{B} + (\bm{V}_\perp \times \bm{B})\times \bm{B}] - \nabla p \times \bm{B}\\
    &= qn[\bm{E}\times \bm{B} + \bm{B}(\bm{v}_\perp \cdot \bm{B}) -\bm{v}_\perp \bm{B}^2] - \nabla p \times \bm{B}
\end{align*}
式変形にベクトル解析の公式$\bm{A}\times (\bm{B}\times \bm{C}) = (\bm{A}\cdot \bm{C})\bm{B} - (\bm{A}\times \bm{B})\bm{C}$を用いた。第二項は0になる。これより、

\begin{equation}
  \bm{v}_\perp = \frac{\bm{E}\times \bm{B}}{B^2} - \frac{\nabla p \times \bm{B}}{qnB^2} \equiv \bm{v}_E + \bm{v}_D \label{3.63}
\end{equation}
となる。ここで、
\begin{equation}
  \boxed{\bm{v}_E \equiv \frac{\bm{E}\times \bm{B}}{B^2}} \label{3.64}
\end{equation}
\begin{equation}
  \boxed{\bm{v}_D \equiv -\frac{\nabla p \times \bm{B}}{qnB^2}} \label{3.65}
\end{equation}
と定義した。$\bm{v}_E$はすでに定義した$\bm{E}\times \bm{B}$ドリフトである。$\bm{v}_D$は\textcolor{red}{反磁性ドリフト}と呼ばれる新しいドリフトである。\eqref{3.52}式を用いると、反磁性ドリフトは

\begin{equation}
  \bm{v}_D = \pm \frac{\gamma KT}{eB}\frac{\hat{\bm{z}}\times \nabla n}{n} \label{3.66}
\end{equation}
と書き換えられる。

\begin{figure}[htbp]
  \centering
  \includegraphics[width=0.5\linewidth]{fig3.4.png}
  \caption{円柱プラズマ内の反磁性ドリフト}
  \label{fig3.4}
\end{figure}

このプラズマの物理的説明は図\refeq{fig3.5}に示される。ここには、磁場中を旋回しているイオンの軌道が描かれている。左に向かって密度勾配(あるいはそれを生み出す圧力勾配)があるとすると、任意の固定された体積要素を通り抜けるイオンの数は、上方にいくものより下方にいくものの方が多い。
それゆえ、旋回中心は止まっていても$\nabla n$と$\bm{B}$に対して垂直方向に流体のドリフトが生じる。

\begin{figure}[htbp]
  \centering
  \includegraphics[width=0.6\linewidth]{fig3.5.png}
  \caption{反磁性ドリフトの説明}
  \label{fig3.5}
\end{figure}

イオンと電子は反対方向にドリフトするので、反磁性電流が生じる。$\gamma = Z = 1$とすると

\begin{equation}
  \bm{j}_D = ne(\bm{v}_{Di} - \bm{v}_{De}) = (KT_i + KT_e)\frac{\bm{B}\times \nabla n}{B^2} \label{3.69}
\end{equation}
と書ける。粒子的描像からいえば、もし旋回中心がドリフトしないのであれば電流は測定できないと考えられるが、流体的描像によると、圧力勾配があるといつでも電流$\bm{j}_D$が流れるということである。
実験室での実験が有限の大きさのプラズマで行われることを考えると、これらの2つの視点は矛盾なく認めることができる。プラズマが図\refeq{fig3.6}のような箱の中にあるとしよう。左方では、右方よりもたくさんの粒子があるために正味の電流は下方に流れることになる。このように考えると流体的描像と一致する。
この説明には壁が反射的であることを前提としている。もし壁が吸収性なら、イオンは電子よりも重く、Larmor半径が大きくなるために壁により多く到達し、吸収される。その結果、プラズマ内に電荷の偏りが生じ電場を生み出す。このようになると、新たなドリフトが生じ、反磁性ドリフトだけのシンプルな説明はできなくなる。
実際、反磁性電流は実在する測定可能な電流である。このように、単粒子軌道の描像だけで議論すると、壁の有無などで状況が変わり、とても複雑になる。一方、流体理論では、反磁性ドリフトのような見かけのドリフトでも正しく表現できる。

\begin{figure}[htbp]
  \centering
  \includegraphics[width=0.6\linewidth]{fig3.6.png}
  \caption{限られた形のプラズマ内の粒子のドリフト}
  \label{fig3.6}
\end{figure}

単一粒子像に現れたgrad Bドリフトと曲率ドリフトについてはどうであろうか?まず、曲率ドリフトは流体像にも存在する。なぜなら、磁場曲がり角を移動する流体要素内の全粒子が遠心力を受けるからである。
流体方程式の右辺に、$\bar{F}_{cf} = \overline{nmv^2_\parallel}/R_c = nKT_\parallel/R_c$の項をつけ加えればよい。

一方で、流体においてgrad Bドリフトは存在しない。磁場は粒子に仕事をしないため、いかなる粒子のエネルギーも変化させられないことから、磁場はマクスウェル分布に影響を与えない。したがって、$\bm{B}$が存在する場合においても速度分布$f(\bm{v})$はMaxwell分布のまま変わらない。
もし、非一様磁場$\bm{B}$中で$f(\bm{v})$がMaxwell分布のままであり、密度勾配がない場合、任意の固定された流体要素中での粒子のドリフトは打ち消し合う。この描像は図\refeq{fig3.8}に示される。個々の粒子を見れば、たしかにgrad Bドリフトをしているが、2つの粒子のドリフトはちょうど逆向きになり、打ち消し合う。

非一様電場$\bm{E}$や密度勾配がある場合には、流体像と粒子像を調和させるのは容易ではない。図\refeq{fig3.6}のような簡単な描像は使えない。

\begin{figure}[H]
  \centering
  \includegraphics[width=0.6\linewidth]{fig3.8.png}
  \caption{非一様磁場では、案内中心はドリフトするが、流体要素はドリフトしない。}
  \label{fig3.8}
\end{figure}


\subsection{\texorpdfstring{$\bm{B}$}{B}に平行な流体ドリフト} \label{section3.5}
流体方程式のz成分は

\begin{equation}
  mn\left[\frac{\partial v_z}{\partial t} + (\bm{v}\cdot \nabla)v_z\right] = qnE_z - \frac{\partial p}{\partial z} \label{3.70}
\end{equation}
である。対流項$(\bm{v}\cdot \nabla)v_z$は二次の項なので、しばしば無視することができる。ここでは複雑な議論を避けることにして、単に$v_z$が空間的に一様な場合についてのみ考える。\eqref{3.52}式を用いると、

\begin{equation}
  \frac{\partial v_z}{\partial t} = \frac{q}{m}E_z - \frac{\gamma KT}{mn}\frac{\partial n}{\partial z} \label{3.71}
\end{equation}
となる。この式は、流体が静電的な力と圧力勾配による力を受けて$\bm{B}$に沿って加速されることを示している。\eqref{3.71}式を質量の小さな電子に適用することで、重要な結果が得られる。$m \to 0$,$q=-e$,$\bm{E} = -\nabla \phi$として、

\begin{equation}
  qE_z = e\frac{\partial \phi}{\partial z} = \frac{\gamma KT_e}{n}\frac{\partial n}{\partial z} \label{3.72}
\end{equation}
電子の易動度は非常に大きいので、熱伝導度はほぼ無限大である。それゆえ電子を空間的に等温と仮定すると、$\gamma = 1$。さらに積分を行うことで、
\[
  e\phi = KT_e \in n + C
\]
すなわち

\begin{equation}
  \boxed{n = n_0\exp(e\phi /KT_e)} \label{3.73}
\end{equation}
となる。これはまさしく電子に対するBoltzmanの関係式である。Boltzmanの関係式は平衡状態を表すため、電子が平均的に動かない定常状態であることが必要である。電子に働く静電気力と圧力勾配がつりあうという条件がBoltzmanの関係式を導いたのである。
\eqref{3.73}式は各磁力線ごとに成り立つことに注意してほしい。磁力線に沿っては電子は自由に動けるため、力のつりあいに従ってBoltzmanの関係式が成り立つからである。

\eqref{3.73}式は図\refeq{fig3.9}を使っても説明できる。この図は、プラズマ中に局所的な密度の集中があるときに何が起きるかを示したものである。密度勾配が図の中心方向へあり、$KT$は定数であるとすると、圧力勾配は中心を向く。プラズマは準中性であるから、その勾配は電子流体にもイオン流体にもできる。
電子流体に対する圧力勾配による力$\bm{F}_p$を考えると、それは動きやすい電子のみを中心から遠ざけ、イオンを残す。その結果、生じた電場$\bm{E}$が、$\bm{F}_p$とは逆向きの力$\bm{F}_E$を電子に及ぼす。この$\bm{F}_E$と$\bm{F}_p$とが、大きさが等しく逆向きのときのみ定常状態が成り立つ。
$\bm{B}$が一定であれば、図\refeq{fig3.9}が示すように、$\phi$はnが大きい中心のところで大きくならなければならない。これが\eqref{3.73}式の示すところである。完全な中性からのずれは、自ら調整され、電子にかかる力を均衡させるのに必要な電場を作り出すのにちょうど十分な電荷が生じるようにする。

\begin{figure}[H]
  \centering
  \includegraphics[width=0.6\linewidth]{fig3.9.png}
  \caption{密度とポテンシャル間のBoltzmanの関係に関する物理的説明}
  \label{fig3.9}
\end{figure}


\subsection{プラズマ近似} \label{section3.6}
通常の電磁気学では、$\sigma$をPoisson方程式に代入することで電場$\bm{E}$を求める。しかし プラズマでは逆の過程が使われる。$\bm{E}$は運動方程式から求められ、Poisson方程式は$\sigma$を求めるときのみ使われる。
その理由は、プラズマが常に中性状態を維持しようとする傾向を持つためである。イオンが移動すれば、電子はそれに追随する。$\bm{E}$は、電子とイオンの軌道が中性性を保つように自らを調整する。$\sigma$はPoisson方程式が満たされるように調整されることから、決めるべき主役ではなく、結果として従う量という意味で2次的な重要性しかない。
もちろん、これらのことは、電子がすぐ動いて中性を回復できる低周波運動においてのみ成立する。高周波であれば、電子が追従できず、電荷分離が起こるため、Poisson方程式が必要である。

プラズマにおいては、通常$n_i = n_e$と$\nabla \cdot \bm{E}\neq 0$を同時に満たされると仮定することができる。これを\textcolor{red}{プラズマ近似}と呼び、プラズマの基本的特性である。必要なとき以外は、$\bm{E}$を求めるのにポアソン方程式を使用してはならないことを心掛けてほしい。
流体方程式のセット\eqref{3.55}~\eqref{3.61}式において、Poisson方程式を消すとともに$n_i=n_e=n$とおくことで未知数を1つ消せばいい。

プラズマ近似は、\ref{section1}章に論じた準中性とほぼ同じだが、より厳密な意味を持つ。準中性がプラズマの静止状態における中性傾向を指すのに対し、プラズマ近似は波動運動に対しても適用可能な数学的近似法である。
電子もイオンも十分ゆっくり動ける低周波のときは電荷分離がほぼ起きないので、Poisson方程式を使わなくても$n_i = n_e$と置けばよい。もちろん、高周波電子波のように一方の種のみが移動でき他方が追随できない場合、
プラズマ近似は成立せず、$\bm{E}$はイオンと電子の運動方程式ではなくPoisson方程式から求めなければならない。プラズマ近似の妥当性については、イオン波の理論を扱う際(\ref{section4.5}節)に改めて検討する。その際、Debye遮蔽の導出にポアソン方程式を用いる必要があった理由が明らかになる。


\section{プラズマ中の波動} \label{section4}
\subsection{位相速度・群速度} \label{section4.1}
任意の正弦振動している量は

\begin{equation}
  n = \tilde{n}\exp[i(\bm{k}\cdot \bm{r} - \omega t)] \label{4.1}
\end{equation}
と表せる。もし波がx方向に伝播するとすれば、\eqref{4.1}式は次のようになる。
\[
   n = \tilde{n}e^{i(kx - \omega t)}
\]
波の位相が一定となる点は、$(d/dt)(kx-\omega t)=0$となるように動いていく。

\begin{equation}
  \boxed{\frac{dx}{dt} = \frac{\omega}{k} \equiv v_\phi} \label{4.4}
\end{equation}
これを\textcolor{red}{位相速度}という。

プラズマ中の波の位相速度はしばしば光速$c$を超える。しかし、振幅が一定で無限に長い波列は情報を運ぶことができないので相対性理論に反しない。波のうねりが情報を担い、

\begin{equation}
  \boxed{v_g = \frac{d\omega}{dk}} \label{4.10}
\end{equation}
で伝播する。これを\textcolor{red}{群速度}という。群速度は常に$c$より小さい。


\subsection{プラズマ振動} \label{section4.3}
プラズマ中の電子が均一なイオン背景から変位すると、プラズマの中性を回復させる方向に電場が作られる。この電場により電子は元の位置に引き戻されるが、電子は軽いので慣性により行き過ぎてしまう。そしてまた反対方向に引かれる。
この繰り返しによって、電子は平衡位置のまわりで振動することになる。この振動数は\textcolor{red}{プラズマ振動数}と呼ばれる。プラズマ振動は非常に高速であるため、質量の大きいイオンは振動する電場に反応する時間がなく、固定されていると見なすことができる。
図\refeq{fig4.2}は電荷密度の変化を模式的に示したものであり、電子が不足して正に帯電した領域(開いた長方形)と電子が過剰な負に帯電した領域(塗りつぶされた長方形)が現れている。先ほどの述べたようにこれが空間的に周期的な電場を引き起こし、電子を中性位置に戻そうとする。

最も単純な場合におけるプラズマ周波数$\omega_p$を求めよう。そのためには以下の仮定を設ける: (1)磁場が存在しない;(2)熱運動がない($KT=0$);(3)イオンが空間内で均一で固定されている;(4)プラズマが無限に広がっている;(5)電子運動がx方向のみに生じる。
最後の仮定の結果として、次の式が得られる:

\begin{equation}
  \bm{E} = E\hat{\bm{x}}\hspace{2pc} \nabla \times \bm{E} = 0 \hspace{2pc} \bm{E} = -\frac{d\phi}{dx} \label{4.11}
\end{equation}
変動する磁場は存在せず、これは静電振動である。

電子の運動方程式と連続方程式は
\begin{equation}
  mn_e\left[\frac{\partial \bm{v}_e}{\partial t} + (\bm{v}_e \cdot \nabla)\bm{v}_e\right] = -en_e\bm{E} \label{4.12}
\end{equation}
\begin{equation}
  \frac{\partial n_e}{\partial t} + \nabla \cdot (n_e\bm{v}_e) = 0 \label{4.13}
\end{equation}
となる。Maxwell方程式で必要なものは、$\bm{B}$を含まないPoisson方程式である。このケースは、高周波振動であるため、\ref{section3.6}節の一般則(ポアソン方程式を用いて$\bm{E}$を求めることはできない)に対する例外である。その結果、

\begin{equation}
  \epsilon_0 \nabla \cdot \bm{E} = \epsilon \partial \bm{E}/ \partial x = e(n_i - n_e) \label{4.14}
\end{equation}
\eqref{4.12}~\eqref{4.14}式は線形化することで簡単に解ける。まず、それぞれを下付き添字0で示される「平衡」部分と、下付き添字1で示される「摂動」部分の2つに分ける:

\begin{equation}
  n_e = n_{0} + n_{1} \hspace{2pc} \bm{v}_e = \bm{v}_{0} + \bm{v}_{1} \hspace{2pc} \bm{E} = \bm{E}_0 + \bm{E}_{1} \label{4.15}
\end{equation}
添字0平衡部分は振動のないときのプラズマの状態を表すが、ここでは電子が変位する前は静止した均一な中性プラズマであると仮定しているため、


\begin{align}
  \begin{split}
    \nabla n_0 = \bm{v}_0 = \bm{E}_0 = 0\\
    \frac{\partial n_0}{\partial t} = \frac{\partial \bm{v}_0}{\partial t} = \frac{\partial \bm{E}_0}{\partial t} = 0
  \end{split} \label{4.16}
\end{align}
となる。\eqref{4.12}式は、これより

\begin{equation}
  m\left[\frac{\partial \bm{v}_1}{\partial t} + (\bm{v}_1 \cdot \nabla)\bm{v}_1\right] = -e\bm{E}_1 \label{4.17}
\end{equation}
































\[
\hat{\mu}_1
= g\mu_0
\begin{pmatrix}
\beta & 0 & 0 \\
0 & 1 & 0 \\
0 & 0 & 1
\end{pmatrix}
\]

\[
\hat{\mu}_2
= g\mu_0
\begin{pmatrix}
1 & 0 & 0 \\
0 & \beta & 0 \\
0 & 0 & 1
\end{pmatrix}
\]

\[
\hat{\mu}_3
= g\mu_0
\begin{pmatrix}
1 & 0 & 0 \\
0 & 1 & 0 \\
0 & 0 & \beta
\end{pmatrix}
\]

\[
\hat{\mu}_5
= \frac{\mu_0}{3}
\begin{pmatrix}
\beta+2 & \beta-1 & 1-\beta \\
\beta-1 & \beta+2 & 1-\beta \\
1-\beta & 1-\beta & \beta+2
\end{pmatrix}
\]

\[
\hat{\mu}_6
= \frac{\mu_0}{3}
\begin{pmatrix}
\beta+2 & 1-\beta & 1-\beta \\
1-\beta & \beta+2 & \beta-1 \\
1-\beta & \beta-1 & \beta+2
\end{pmatrix}
\]

\[
\hat{\mu}_7
= \frac{\mu_0}{3}
\begin{pmatrix}
\beta+2 & 1-\beta & \beta-1 \\
1-\beta & \beta+2 & 1-\beta \\
\beta -1 & 1-\beta & \beta+2
\end{pmatrix}
\]

\[
  \gamma = \frac{1-\beta}{3}
\]








\section{結論}
LaTeXを使えば、美しい日本語の文書を作成することができます。ぜひ活用してみてください。




\begin{thebibliography}{99}
\bibitem{a} Francis F. Chen, Introduction to Plasma Physics and Controlled Fusion, Springer; Softcover reprint of the original 3rd ed., 2019\par
\url{http://library.unisel.edu.my/equip-unisel/custom/ebook_catalog/2016BookIntroductionToPlasmaPhysicsAndcompressed.pdf}
\bibitem{b} 宮本健郎, プラズマ物理の基礎, 朝倉書店 , 2014\par
\url{https://op.lib.kobe-u.ac.jp/opac/opac_details/?reqCode=fromlist&lang=0&amode=11&bibid=2002133568&opkey=B175679900718865&start=1&totalnum=4&listnum=1&place=&list_disp=20&list_sort=0&cmode=0&chk_st=0&check=0000}
\bibitem{c} EMANの物理学\par
\url{https://eman-physics.net/electromag/plasma.html}
\end{thebibliography}





\end{document}