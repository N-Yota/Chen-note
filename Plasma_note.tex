\documentclass[a4paper,12pt]{article}

% 日本語対応(Macなのでluatexjaを推奨)
\usepackage{luatexja}
\usepackage[ipaex]{luatexja-preset} % IPAexフォントを利用

% 数式や図表は普通に使える
\usepackage{amsmath, amssymb}
\usepackage{graphicx}

\title{Introduction to Plasma Physics and Controlled Fusion}
\author{中 陽太}
\date{\today}

\begin{document}

\maketitle
\section{Introduction}
\subsection{}
本レポートでは、プラズマ物理学における基礎的な概念について説明する。
特に、デバイ遮蔽やプラズマ振動といった現象を中心に議論する。

\section{数式の例}
たとえば、プラズマ振動数は以下で与えられる:

\[
\omega_p = \sqrt{\frac{n e^2}{\epsilon_0 m_e}}.
\]

\section{結論}
日本語で本文を書きつつ、数式や参考文献は通常の LaTeX 形式を利用できる。

\end{document}
